\documentclass[12pt]{scrartcl}%{article} % Beginn der LaTeX-Datei
\usepackage{mathtools,amsmath,amssymb,amsthm}  % erleichtert Mathe 
\addtolength{\jot}{0.2em}
\usepackage{enumitem}% schicke Nummerierung
\newcommand{\sbt}{\,\begin{picture}(-1,1)(-1,-3)\circle*{3}\end{picture}\ }
\renewcommand{\labelitemi}{\sbt}

\usepackage{graphicx} % für Grafik-Einbindung
\usepackage{float} 
%\usepackage[dvips]{hyperref}
\usepackage{tikz-cd}
\tikzset{ % das braucht man für die kommutativen Diagramme, wenn man babel german benutzt!!
  every picture/.append style={
    execute at begin picture={\shorthandoff{"}},
    execute at end picture={\shorthandon{"}}
  }
}
\usepackage{nicefrac}

% environments
\newtheorem{thm}{Theorem}
\newtheorem{lemma}{Lemma}
\newtheorem{kor}{Korollar}
% definition-like stuff
\theoremstyle{definition}
\newtheorem*{defn}{Definition}
\newtheorem{ex}{Beispiel}
% remark-like stuff
\theoremstyle{remark}
\newtheorem*{notation}{Notation}
\newtheorem*{nb}{Bemerkung}

% commands
\newcommand{\powerset}{\mathcal{P}}
\newcommand{\sym}{\text{Sym}}
\newcommand{\gl}{\text{GL}}
\newcommand{\abb}{\text{Abb}}
\newcommand{\inv}[1]{\left(#1\right)^{-1}}
\newcommand{\Inv}[1]{#1^{-1}}

\usepackage[ngerman]{babel}
\usepackage[autostyle=true,german=quotes]{csquotes}
%\usepackage[T1]{fontenc}
%\usepackage{lmodern}
% Einstellungen, wenn man deutsch schreiben will, z.B. Trennregeln
%\usepackage[applemac]{inputenc}
\usepackage[utf8]{inputenc}  % für Unix-Systeme

% TODO Bspw. ord wird zZ als drei vAriablen (o, r, d) gesetzt. Zu einem vernünftigen Operator ändern?
% TODO Anführungszeichen sind bei mir alle falsch. Funktioniert das bei euch?
% TODO Einige Beweise sind schrecklich strukturiert und zum Teil rettbar fehlerhaft. Darf ich?

\begin{document}

\author{Sebastian Bechtel, Isburg Knof, Theresa Tran}
\title{Einführung in die Algebra}
\date{15. April 2015}

\maketitle

\section{Gruppen}

\begin{defn}
    Eine (innere) \underline{Verknüpfung} auf einer Menge $M\neq \emptyset$ ist eine Abbildung $M\times M\to M, (a,b)\mapsto a\cdot b$.
\end{defn}

\begin{defn}
    Eine \underline{Gruppe} ist eine Menge $G\neq \emptyset$ zusammen mit einer Verknüpfung $\cdot$, sodass Assoziativität (A), Existenz eines neutralen Elements (N) und Existenz inverser Elemente (I) erfüllt sind. $G$ ist \underline{abelsch}, falls Kommutativität (K) gilt.
\end{defn}

\begin{ex}
    \begin{enumerate}
        \item $\mathbb{Z}, \mathbb{Q}, \mathbb{R}, \mathbb{C}$ sind abelsche Gruppen mit $+$ als Verknüpfung.
        \item $\mathbb{Q}^*=\mathbb{Q}\setminus \{0\}, \mathbb{R}^*, \mathbb{C}^*$ mit Multiplikation sind abelsche Gruppen.
        \item Für eine Menge $M$ ist $\sym(M)$ ist eine Gruppe, aber nicht abelsch.
    \end{enumerate}
\end{ex}

\begin{lemma}
    \begin{enumerate}[label=\alph*)]
        \item Das neutrale Element ist eindeutig.
        \item Inverse Elemente sind eindeutig.
    \end{enumerate}
\end{lemma}

\begin{proof}
    \begin{enumerate}[label=\alph*)]
        \item Seien $e,f$ neutrale Elemente, dann gilt $e=ef=f$.
        \item Sei $a\in G$ und $b,b'\in G$ inverse Elemente. Dann gilt $b'=b'e=b'(ab)=(b'a)b=eb=b$. 
    \end{enumerate}
\end{proof}

\begin{notation}
    multiplikativ: $a\cdot b$ oder $ab$, neutrales Element $e$ oder $1$, inverses Element von $a\in G$ ist $\Inv a$.
\end{notation}

\begin{lemma}
    Es sei $\mathcal{G}=(G,\cdot)$ eine Menge mit assoziativer Verknüpfung, einem linksneutralen Element und linksinversen Elementen, dann ist $\mathcal{G}$ eine Gruppe.
\end{lemma}

\begin{proof}
    Sei $a\in G$ und $b\in G$ mit $ba=e$. Nach (I') gibt es $c\in G$ mit $cb=e$. Also $ab=eab=cbab=ceb=cb=e$.

    Sei nun $a\in G$, es gilt $ae=a(\Inv aa)=ea=a$.
\end{proof}

\begin{lemma}
    \begin{enumerate}
        \item $\inv{\Inv{a}}=a$, $\inv{ab}=\Inv b\Inv a$
        \item $ab=ac$ impliziert $b=c$ für alle $a,b,c\in G$.
        \item für $a,b\in G$ gibt es genau ein $x\in G$, sodass $ax=b$.
    \end{enumerate}
\end{lemma}

\begin{proof}
    \begin{enumerate}
        \item $\inv{\Inv a}=a$ klar. Für $a,b\in G$: $(\Inv b\Inv a)ab=\Inv b(\Inv aa)b=\Inv beb=\Inv bb=e$ (andere Richtung analog) % FIXME Welche andere Richtung?
        \item $ab=ac$ impliziert $\Inv a(ab)=\Inv a(ac)$ impliziert $b=c$
        \item Setze $x=\Inv ab$, dann erhält man $ax=a(\Inv ab)=(a\Inv a)b=eb=b$.
       	Die Eindeutigkeit folgt aus Punkt 2. \qedhere
    \end{enumerate}
\end{proof}

\begin{defn}
    Sei $a\in G$, $(G,\cdot)$ Gruppe. Für $n\in \mathbb{Z}$ definiere:
    
    $$a^0:=e, \quad a^n:=a^{n-1}a \quad \text{ für } n\geq 1$$
    $$a^n:=\left(\Inv a\right)^{-n} \quad \text{ für } n < 0$$
\end{defn}

\begin{lemma}
    Für $a\in G$ gelten $a^n a^m=a^{n+m}=a^m a^n$ und $\left(a^m \right)^n = a^{n\cdot m}$.
    $ab=ba$ impliziert $\left(ab \right)^n = a^n b^n$.
\end{lemma}

\begin{ex}
    \begin{enumerate}
        \item $K$ Körper, dann ist $\gl_n(K)$ ein Gruppe bzgl. Matrixmultiplikation.
        \item $M\neq \emptyset$ Menge, $(G, \cdot)$ Gruppe, definiere $\abb(M,G):=G^M$. Für $f,g\in \abb(M,G)$ ist $f\cdot g$ gegeben durch $(f\cdot g)(m)=f(m)\cdot g(m)$ für $m\in M$. Dann ist $(\abb(M,G), \cdot)$ eine Gruppe.
    \end{enumerate}
\end{ex}



\section{Untergruppen}

\begin{defn}
    Sei $(G, \cdot)$ Gruppe. Eine Teilmenge $H\subset G$ heißt Untergruppe von $G$, falls $(H, \cdot)$ eine Gruppe ist.

    Äquivalent dazu:

    \begin{enumerate}[label=(\roman*)]
        \item Für $a,b\in H$ gilt $ab\in H$ (Abgeschlossenheit)
        \item $e\in H$
        \item Für $a\in H$ ist $\Inv a \in H$
    \end{enumerate}
\end{defn}

\begin{thm}
    Sei $(G, \cdot)$ Gruppe und $H\subset G$ nicht-leer. Dann gilt: $H$ induziert Untergruppe von $(G, \cdot)$ gdw. $a\Inv b\in H$ für $a,b\in H$.
\end{thm}

\begin{proof}
    "$\Rightarrow$" \checkmark

    "$\Leftarrow$" 
    \begin{itemize} 
        \item $a=b$ impliziert $e\in H$
        \item $e,a\in H$ impliziert $e\Inv a\in H$ impliziert $\Inv a \in H$
        \item $a,\Inv b\in H$ impliziert $a\inv{\Inv b} \in H$ impliziert $ab\in H$ \qedhere
    \end{itemize}
\end{proof}

\begin{ex}
    \begin{enumerate}[label=(\alph*)]
        \item $\{e\}, G$ induzieren je eine Untergruppe für alle Gruppen $(G,\cdot)$.
        \item $K$ Körper. $\text{SL}_n(K)=\{A\in M_n(K): \det(A)=1\}$ induziert Untergruppe von $\gl_n(K)$, die spezielle lineare Gruppe.
    \end{enumerate}
\end{ex}

\begin{defn}
    Eine Untergruppe heißt \underline{echt}, falls sie nicht trivial ist. % FIXME Welche sind trivial? {e} und G?
\end{defn}

\begin{lemma}
	Es sei $(H_{j})_{j \in J}$ eine Familie von Untergruppen $H_{j} \subset G$. Dann ist $\bigcap_{j \in J}H_{j}$ eine Untergruppe von G.
\end{lemma}

\begin{proof}
	Übung
\end{proof}

\begin{defn}
	Es sei $M$ eine Teilmenge von $G$. Die \underline{von $M$ erzeugte Untergruppe} ist der Durchschnitt aller Untergruppen, die $M$ enthalten.
\end{defn}

\begin{notation}
	$\langle M \rangle =\bigcap_{M \subset H \subset G}H$, wobei $H$ Untergruppe
\end{notation}

\begin{nb}
	\begin{enumerate}[label=(\alph*)]
		\item $\langle \emptyset \rangle = \lbrace e \rbrace$
		\item Für $M \neq \emptyset$ gilt: $\langle M \rangle = \lbrace m_{1}^{\varepsilon_{1}} \cdot ... \cdot m_{n}^{\varepsilon_{n}} : m_{1},...,m_{n} \in M, \varepsilon_{1},...,\varepsilon_{n} \in \lbrace -1,+1\rbrace, n \geq 0 \rbrace$
		\item Für $M = \lbrace g \rbrace$ gilt: $\langle g \rangle = \lbrace g^{n} : n \in \mathbb{Z} \rbrace$. Von g erzeugte zyklische Untergruppe von G.
	\end{enumerate}
\end{nb}

\begin{defn}
	$G$ heißt \underline{zyklisch}, falls $G = \langle g \rangle$ für ein $g \in G$ gilt. \newline
	Ist $G = \langle M \rangle$ mit $M$ endlich, so heißt $G$ \underline{endlich erzeugt}.
\end{defn}

\begin{defn}
	\begin{enumerate}[label=(\roman*)]
		\item Die \underline{Ordnung einer Gruppe} $G$ ist $ord(G)=\vert G \vert$.
		\item Die \underline{Ordnung eines Elements} $g \in G$ ist $ord(g)=ord(\langle g \rangle)$.
		\item Ist $ord(g)$ endlich, dann hat g \underline{endliche Ordnung}.
	\end{enumerate}
\end{defn}

\begin{notation}
    $(n,s)$ bezeichnet den größten gemeinsamen Teiler.
\end{notation}

\begin{thm}
	Sei $G$ Gruppe, $g \in G$
	\begin{enumerate}
		\item g hat nicht endliche Ordnung $\Longleftrightarrow$ alle Potenzen von g sind verschieden
		\item g hat endliche Ordnung $\Longleftrightarrow$ $\exists m>0: g^{m}=e$ \newline Dann gilt:
			\begin{enumerate}[label=(\alph*)]
				\item $n := ord(g) = \min \lbrace m>0 : g^{m}=e \rbrace$
				\item $g^{m}=e \Longleftrightarrow m=nk$ für ein $k \in \mathbb{Z}$
				\item $\langle g \rangle = \lbrace e, g^{1},...,g^{n-1}\rbrace$
			\end{enumerate}
		\item $ord(g^{s}) = \frac{n}{(n,s)}$ für $n = ord(g)$ endlich
	\end{enumerate}
\end{thm}

\begin{proof}
	\begin{enumerate}
		\item Wir nehmen an: Für $i,j \in \mathbb{Z}$, oBdA $j>i$ gilt $g^{i}=g^{j}$.
		Dann gilt $g^{j-i}=g^{j}(g^{i})^{-1}=e$.
		Es sei dann n die kleinste positive Zahl, die $g^{n}=e$ erfüllt.
		Sei $m \in \mathbb{Z}$ beliebig.
		Der Divisionsalgorithmus liefert: $m=kn+r$ für $0 \leq r < n$ und $k,r \in \mathbb{Z}$.

		Dann gilt:
		$g^{m}=g^{kn+r}=g^{kn}g^{r}=(g^{n})^{k}g^{r}=eg^{r}=g^{r}$.
		Daraus folgt $\langle g \rangle = \lbrace g^{m} : m \in \mathbb{Z}\rbrace = \lbrace g^{r} : r=0,...,n-1\rbrace$.
		Besonders gilt $ord(g)=n$ ist endlich.
		$\lceil$ Dies zeigt $\Rightarrow$, $\Leftarrow$ klar, dann ist $\langle g \rangle = \lbrace g^{m} : m \in \mathbb{Z}\rbrace$ unendlich $\rfloor$
		\item Alle $g^{r}$ mit $0 \leq r \leq n-1$ sind verschieden, da:
		$g^{i}=g^{j} \Rightarrow g^{j-i}=e \Rightarrow j-i = kn$ mit $k\in\mathbb{Z} \Rightarrow i \equiv j \pmod{n} \Rightarrow i=j$ falls $0\leq i,j\leq n-1$.
		Dies liefert $g^{r}$ mit $0 \leq r \leq n-1$ sind paarweise verschieden und es gilt: $ord(g)=n$. $\lceil a$ und $c\rfloor$
		Aus dem Divisionsalgorithmus folgt (b): $g^{m}=e \Leftrightarrow e=g^{kn+r}=g^{r}$ mit $m = kn+r, 0 \leq r < n \Leftrightarrow r=0$.
		Also $m=kn$ mit $k \in \mathbb{Z}$. % FIXME Scoping-Katastrophe.
		\item Es sei $m=ord(g^{s}), n =ord(g)$.
		Aus $(g^{s})^{m}=e$ folgt (siehe 2), dass $n$ ein Teiler von $sm$ ist.
		Dies liefert: $\frac{n}{(s,n)} \vert \frac{s}{(s,n)}m$. Somit $\frac{n}{(s,n)} \vert m$.
		Nun möchten wir noch zeigen: $m \vert \frac{n}{(s,n)}$. $(g^{s})^{\frac{n}{(s,n)}} = (g^{n})^{\frac{s}{(s,n)}}=e^{\frac{s}{(s,n)}}=e$.
		Daraus folgt $m \vert \frac{n}{(s,n)}$ (wegen 2).
		Also gilt $m = \frac{n}{(s,n)}$. \qedhere
	\end{enumerate}
\end{proof}

\begin{lemma}
	Wir können alle Untergruppen einer zyklischen Gruppe beschreiben mit $G = \langle g \rangle, H \subset G$, es sei $h \in H, h \neq e$. Dann gilt: $h = g^{k}$. % FIXME Aussage des Lemmas unverständlich. Kann den Beweis nicht überprüfen.
\end{lemma}

\begin{proof}
	Wir setzen: $m = min \lbrace k>0 : g^{k} \in H \rbrace$. \newline $\lceil$ Existiert: $G= \langle g \rangle = \langle g^{-1} \rangle, h = g^{k}, k<0$, dann ersetzen wir $h$ durch $h^{-1} \rfloor$ \newline Wir wollen zeigen: $\langle g^{m} \rangle = H$ 
	\begin{enumerate}
		\item $\langle g^{m} \rangle \subset H$ gilt wegen $g^{m} \in H$
		\item Es sei $j \in  \mathbb{Z}$ mit $g^{j} \in H$. Divisionsalgorithmus liefert $j=lm+r$ mit $0 \leq r < m$: $g^{j} \in H \Rightarrow g^{r}=g^{-lm}g^{lm+r}=(g^{m})^{-l}g^{j}$. Also $g^{r} \in H$. Aus der Minimalität von M folgt $r=0$. Dies liefert $g^{j}=(g^{m})^{l} \in \langle g^{m} \rangle$ und somit gilt: $H \subset \langle g^{m} \rangle$ und die zwei Untergruppen stimmen überein.
	\end{enumerate}
\end{proof}

Ähnlich kann man zeigen:

\begin{thm}
	Alle Untergruppen einer zyklischen Gruppe sind zyklisch. Ist $ord(G)=n$ endlich und $m$ ein Teiler von $n$, so ist $H = \langle g^{\frac{n}{m}}\rangle$ die einzige Untergruppe der Ordnung $m$. % FIXME Beweis?
\end{thm}

\begin{defn}
	Sei $H$ eine Untergruppe der Gruppe $G$. Dann kann man die folgende Äquivalenzrelation definieren: \newline $(x,y) \in G^{2}: x \sim_{H} y \Leftrightarrow x = yh$ für ein $h \in H$ \newline $\lceil$Äquivalenzrelation wegen Gruppenaxiomen für $H\rfloor$
\end{defn}

\begin{defn}
	Die Äquivalenzklassen bzgl. $\sim_{H}$ heißen \underline{Linksnebenklassen}.
\end{defn}

\begin{notation}
	Für $a \in G, aH = \{ah : h \in H\}$
\end{notation}

\begin{nb}
	Es gelten folgende Eigenschaften: 
	\begin{itemize}
		\item Die Abbildung $H \rightarrow aH, h \mapsto ah$ ist eine Bijektion. Besonders gilt: $\vert aH \vert = \vert H \vert$ für alle $a \in G$. \newline $\lceil$Die Abbildung ist bijektiv, da sie umkehrbar ist: $aH \rightarrow H, b \mapsto a^{-1}b$ ist die Umkehrfunktion$\rfloor$
		\item $aH \neq bH \Rightarrow aH \cap bH = \emptyset$, d.h. sie sind disjunkt. \newline $\lceil x \in aH \neq bH \Rightarrow x = ah_{1} = bh_{2}$ für $h_{1},h_{2} \in H \Rightarrow a=bh_{2}h_{1}^{-1} \in bH \Rightarrow ah= b(h_{2}h_{1}^{-1}h) \in bH$ für alle $h \in H \Rightarrow aH \subset bH$. Ähnlich gilt $bH \subset aH$. Daraus folgt $aH=bH.\rfloor$ % TODO Äquivalenzklassen sind immer disjunkt. Wieso der Aufwand?
	\end{itemize}
\end{nb}

\begin{defn}
	$G/H = \lbrace aH : a \in G \rbrace$ ist die \underline{Menge der Linksnebenklassen}. \newline Der Index von $H$ ist die Mächtigkeit von $G/H$, d.h. \underline{Index} [$G:H$]$:=\vert G/H \vert$
\end{defn}

\begin{nb}
	\begin{itemize}
	 	\item $\vert G \vert = [G:H]\vert H\vert$
	 	\item Analog ist $a \sim_{H} b$ mit $a,b \in G \Leftrightarrow a=hb$ für ein $h \in H$ ("rechtsäquivalent bzgl. H") eine Äquivalenzrelation.
		\underline{Rechtsnebenklassen}: $Ha=\lbrace ha : h \in H\rbrace$ mit $a \in G$ \newline Bijektion: Für $a \in G$ $aH \rightarrow Ha, x \mapsto a^{-1}xa$
	\end{itemize}
\end{nb}

\begin{defn}
	$H \backslash G$ ist die \underline{Menge der Rechtsnebenklassen}. Dann gilt: $\vert H \backslash G \vert = \vert G/H \vert$ \newline $\lceil$Bijektion: $H \backslash G \rightarrow G/H, Hb \mapsto b^{-1}H \rfloor$
\end{defn}

% Hier fehlt die Vorlesung vom 28.April. Viel Spaß beim texen, Philipp :D
\begin{lemma} 
Die Funktion
$$  	H\backslash G   \xrightarrow{f} G/H $$
$$	Hb \rightarrow b^{-1}H$$
ist eine Bijektion

\end{lemma}

\begin{proof}
1) f ist wohldefiniert:\\
Gilt $Hb_{1}=Hb_{2}$ für $b_{1},b_{2} \in G$
das heißt $b_{1} {_{\ H}}\sim b_{2}$ \\
existiert ein $h \in H$ mit $b_1 = hb_2$ \\
$b^{-1}*H = (hb_2)^{-1}*H = b_{2}^{-1}*h*H =  b_{2}^{-1}*H$ 

2) f ist Bijektion, da sie wohldefiniert ist:\\
$$  	G/H   \xrightarrow{g} H\backslash G $$
$$	Ha \rightarrow Ha^{-1}$$
ist wohldefiniert, ähnlich zu (1).\\
$g \circ f = Id_{H \backslash G}$ weil $G \circ f(Hb) = gb^{-1}H=H(b^{-1})^{-1}=Hb$ \\
$f \circ g = Id_{G \slash H}$ analog.
\end{proof}

\begin{theorem}
Es seien $H$,$K$ Untergruppen von $G$ mit $K \subset H \subset G$. Dann gilt: $$[G:K]=[H:K]*[G*H]$$
\end{theorem}

\begin{proof}
Wir nehmen an, dass $[G:H]=m$ und $[H:K]=n$ endlich sind. 
Dann gibt es $a_1,.....,a_n \in H$, so dass
$H/K=\{ a_1K,...,a_nK\}$ und $b_1,...,b_m \in G$ mit
$G\backslash H=\{ b_1H,...,b_mH\}$

Dann gilt:
$$G=\bigcup\limits_{\begin{subarray}{1}i=1...m \\ j=1...n\end{subarray}} b_ia_jK$$
mit $b_ia_jK$ paarweise disunkt.

$b_{i_{1}}a_{j_{1}}K=b_{i_{2}}a_{j_{2}}K$ mit $i_1,i_2 \in \{1,...,n \}$ und $j_1,j_2 \in \{1,...,m \}$

$\Rightarrow b_{i_{1}}H$ und $b_{i_{2}}H$ sind nicht disjunkt.

$\Rightarrow b_{i_{1}}H=b_{i_{2}}H$

Dann gilt: $b_{i_{1}}a_{j_{1}}K=b_{i_{1}}a_{j_{2}}K$

$\overset{*b^{-1}}{\Rightarrow} a_{j_{1}}K=a_{j_{2}}K \Rightarrow a_{j_{1}}=a_{j_{2}}$ \\
Also: Es gibt genau $m*n$ Linksnebenklassen von $K$ in $G$.

$\bigcup\limits_{\begin{subarray}{1}i=1...m \\ j=1...n\end{subarray}} b_ia_jK =  \bigcup\limits_i b_i( \bigcup\limits_j a_jK)= \bigcup\limits_i b_iH=G$

\end{proof}

\begin{remark}
Der Beweis lässt sich zum Fall von unendlichem Index erweitern. Dies liefert eine Bijektion $G/H \times H/K \to G/K$
\end{remark}

\begin{corollary}{Satz von Lagrange}\\
Es gilt $\vert G\vert = [G:H]*\vert H\vert$ für jede Untergruppe  $H$ von $G$. Besonders gilt: $\vert H\vert$ teilt $\vert G\vert$ und $ord(g)$ ist ein Teiler von  $\vert G\vert$ für jedes $g\in G$.
\end{corollary}

\begin{proof}
$K= \{ e\}$ im vorigen Satz.
\end{proof}

\begin{example}
$G$ endlich mit $\vert G\vert=p$ Primzahl. Dann existiert ein $g \in G$ mit $g \neq e$

$ord(g)=p \Rightarrow <g> besteht aus genau p Elementen.$\\
Also $<g>=G$ und $G$ ist die von $g$ erzeugte zyklische Gruppe.
\end{example}

\section{Normalen Untergruppen und Gruppenhomomorphismen}

\begin{theorem}
Es sei $G$ eine Gruppe unf $H\in G$ eine Untergruppe.

Die folgenden Bedingungen sind äquivalent:

(i) es gilt: $bH=Hb$ für alle $b\in G$
(ii) es gilt: $b^{-1}Hb=H$ für alle $b\in G$
(iii) es gilt: $b^{-1}hb=H$ für alle $b\in G$ und $h\in H$
\end{theorem}

\begin{definition}
Eine Untergruppe $H$, die eine der Bedingungen (i)-(iii) erfüllt, nennt man eine normale Untergruppe (oder Normalteiler) von $G$.
\end{definition}

% TODO Falls, was auch immer Philipp vom TeXen abhält, nicht behoben ist, sollte diese Vorlesung so langsam neu vergeben werden.

\newpage
\thispagestyle{empty}
Hier fehlt die Vorlesung vom 28.April. Viel Spaß beim \TeX en, Philipp :D

\newpage





\begin{defn}
	A, B, C Gruppen, $ f: G \rightarrow H$ Gruppenhomomorphismus. \\
	Das \underline{Bild von $f$} ist: $im(f) = \{f(g) | g \in G\} \subseteq H$. \\
	Der \underline{Kern von $f$} ist: $ker(f) = \{g \in G | f(g) = e_H\} \subseteq G$.
\end{defn}

\begin{lemma}
	Für jeden Gruppenhomomorphismus $ f: G \rightarrow H$ gilt:
	\begin{enumerate}[label=(\roman*)]
		\item $im(f)$ ist eine Untergruppe von H.
		\item $ker(f)$ ist eine normale Untergruppe von G.
		\item $f$ injektiv (bzw. surjektiv) $ \Leftrightarrow ker(f) = {e_G}$ (bzw. $im(f) = H$).
	\end{enumerate}
\end{lemma}

\begin{proof}
	\begin{enumerate}[label=(\roman*)]
		\item $im(f)$ ist wegen der Definition von Gruppenhomomorphismen unter $\cdot_H$ abgeschlossen, enthält $e_H = f(e_G)$ (Eigenschaft (i)) und die Inversen aller seiner Elemente (Eigenschaft (ii)).
		\item Für alle $a, b \in ker(f)$ gilt:
			$$ f(ab^{-1}) = f(a)f(b^{-1}) = f(a)(f(b))^{-1} = ee^{-1} = e.$$
			Somit ist $ker(f)$ eine Untergruppe. \\
			Für jedes $ g \in ker(f)$ und $x \in G$ gilt:
			$$ f(x^{-1}gx) = f(x^{-1})f(g)f(x) = f(x^{-1}) e f(x) = f(x^{-1})f(x) = f(x^{-1}x) = f(e) = e.$$
			Also ist $x^{-1}gx \in ker(f) $ und somit ist $ker(f)$ ein Normalteiler.
		\item Subjektivität: offentsichtlich. Injektivität: vgl. Übung 2, Aufgabe 1.
	\end{enumerate}	
\end{proof}

\begin{nb}
	\begin{itemize}
		\item Ein bijektiver Homomorphismus wird \underline{Isomorphismus} genannt. Die Umkehrfunktion ist dann wieder ein Isomorphismus.
		\item Ein Gruppenhomomorphismus $G \rightarrow G$ heißt \underline{Endomorphismus von G}.
		\item Ein Gruppenisomorphismus $G \rightarrow G$ heißt \underline{Endomorphismus von G}.
	\end{itemize}
\end{nb}

\begin{defn}
	$G/N$ heißt \underline{Faktorgruppe von $N$ in $G$}, wenn $N$ normale Untergruppe von $G$ ist. \\
	Gruppenstruktur: $aN \cdot bN = abN$, $eN = N$ neutrales Element \\
	Die \underline{natürliche Projektion} $\pi : G \rightarrow G/N, a \mapsto aN$ ist ein surjektiver Gruppenhomomorphismus mit $ker(\pi) = N, im(\pi) = G/N$.
\end{defn}

\begin{thm}[Homomorphiesatz]
	Es sei $f: G \rightarrow G'$ ein Gruppenhomomorphismus und N ein Normalteiler von G. Ist N im $ker(f)$ enthalten, so gibt es genau einen Gruppenhomomorphismus $\bar{f}: G/N \rightarrow G'$, so dass das folgende Diagramm kommutiert:

    \[ \begin{tikzcd}
            G \arrow[rr,"f=\bar f\circ \pi"] \arrow[rd,"\pi"] & & G'  \\
                                             & G/N \arrow[ru,"\bar f"] &
    \end{tikzcd} \]
\end{thm}

\begin{nb}
	Es gilt $ker(\bar{f}) = \pi(ker(f))$ und $im(\bar{f}) = im(f)$. 
\end{nb}

\begin{kor}
	Wenn $f: G \rightarrow G'$ ein surjektiver Homomorphismus ist, dann ist $\bar{f}: G/ker(f) \rightarrow G'$ ein Isomorphismus.
\end{kor}

\begin{proof}[Beweis des Satzes]
	Aus dem kommutativen Diagramm folgt, dass $\bar{f}(aN) = f(a)$ sein sollte für jedes $aN \in G/N$. \\
	Wir zeigen zuerst, dass $\bar{f}$ wohldefiniert ist. Es seien $a, b \in G$ mit $aN = bN$. Dann gilt $a \sim_{N} b$ und also $b^{-1}a \in N$. Daraus folgt
	$$f(a) = f(ea) = f(b(b^{-1}a)) = f(b) f(b^{-1}a) \stackrel{b^{-1}a \in N \subseteq ker(f)}{=} f(b)e = f(b).$$
	Die Abbildung $\bar{f}: G/N \rightarrow G'$ existiert somit und ist offensichtlich ein Gruppenhomomorphismus, denn
	$$\bar{f}(abN) = f(ab) = f(a)f(b) = \bar{f}(aN)\bar{f}(bN)$$ for all $a, b \in G$.
\end{proof}

\begin{proof}[Beweis der Bemerkung]
	\begin{itemize}
		\item $aN \in ker(\bar{f}) \stackrel{Def. ~ von ~ \bar{f}}{\Leftrightarrow} a \in ker(f)\stackrel{Def. ~ von ~ \pi}{\Leftrightarrow} aN \in \pi(ker(f))$
		\item $im(f) = im(\bar{f} \circ \pi) = \bar{f}(im(\pi)) \stackrel{\pi ~ surj.}{=} \bar{f}(G/N) = im(\bar{f})$
	\end{itemize}
\end{proof}

\begin{thm}[1. Isomorphiesatz]
	Es seien $G$ eine Gruppe, $H \subseteq G$ eine Untergruppe und $N$ ein Normalteiler von $G$. Dann ist $NH = \{nh | n \in N, h \in H\}$ Untergruppe von $G$ und $N \cap H$ ein Normalteiler von $H$. Ferner ist $H/(N \cap H) \rightarrow (NH)/N, a(N \cap H) \mapsto aN$ ein Isomorphismus.
\end{thm}

\begin{proof}
	Es seien $n_1, n_2 \in N, h_1, h_2 \in H$. Dann gilt:
	$$ (n_1h_1)(n_2h_2)^{-1} = n_1h_1h_2^{-1}n_2^{-1} = n_1(h_1h_2^{-1}n_2^{-1}) = \dotsb $$
	[Da $N$ normal, gilt $h_1h_2^{-1}N = Nh_1h_2^{-1}$. Es gibt also ein $n_3 \in N$ mit $h_1h_2^{-1}n_2^{-1} = n_3h_1h_2^{-1}$.]
	$$ \dotsb = (n_1n_3)(h_1h_2^{-1}) \in NH.$$
	Daraus folgt, dass $NH$ Untergruppe von $G$ ist. \\
	Nun betrachten wir $f: H \rightarrow (NH)/N, a \mapsto aN = Na$. $f$ ist ein surjektiver Gruppenhomomorphismus mit $ker(f) = N \cap H$. Aus dem Homomorphiesatz (+ Korollar) folgt dann, dass $\bar{f}: H/(N \cap H) \rightarrow (NH)/N$ ein Isomorphismus ist.
\end{proof}

\begin{thm}[2. Isomorphiesatz]
	Es seien $M, N$ normale Untergruppen einer Gruppe $G$. Gilt $N \subseteq M$, so ist $M/N$ eine normale Untergruppe von $G/N$ und die Abbildung $(G/N)/(M/N) \rightarrow G/M, (aN)M/N \mapsto aM$ ist ein Isomorphismus.
\end{thm}

\begin{proof}
	$f: G/N \rightarrow G/M, aN \mapsto aM$ (wohldefiniert wegen $N \subseteq M$) ist surjektiver Gruppenhomomorphismus mit $ker(f) = M/N$. Die Aussage folgt dann aus dem Korollar zum Homomorphiesatz.
\end{proof}

\begin{notation}
	Seien $A, B, C$ Gruppen, $\alpha: A \rightarrow B$, $\beta: B \rightarrow C$ Gruppenhomomorphismen.
	\begin{itemize}
		\item Falls $im(\alpha) = ker(\beta)$, so sagt man, dass die Folge $A \stackrel{\alpha}{\longrightarrow} B \stackrel{\beta}{\longrightarrow} C$ bei $B$ \underline{exakt} ist.
		\item $\alpha$ ist surjektiv, falls $A \stackrel{\alpha}{\longrightarrow} B \longrightarrow  \{e\}$ bei $B$ exakt ist. \\
		\begin{tabular}{p{4.3cm}p{.1cm}p{.3cm}l}
			& $b$ & $\mapsto$ & $e$
		\end{tabular}
		\item $\alpha$ ist injektiv, falls $\{e\} \longrightarrow A \stackrel{\alpha}{\longrightarrow} B$ bei $A$ exakt ist. \\
		\begin{tabular}{p{3.1cm}p{.1cm}p{.3cm}l}
			& $e$ & $\mapsto$ & $e_A$
		\end{tabular}
		\item Notation: $\{e\} =: 1$. Eine exakte Folge der Form
		$$ 1 \longrightarrow A \stackrel{\alpha}{\longrightarrow} B \stackrel{\beta}{\longrightarrow} C \longrightarrow 1$$
		(d.h. $\alpha$ injekiv, $\beta$ surjekiv und $im(\alpha) = ker(\beta)$), heißt \underline{kurze exakte Folge}.
	\end{itemize}
\end{notation}

\begin{ex}[für kurze exakte Folgen]
	\begin{itemize}
		\item $1 \longrightarrow N \stackrel{\alpha}{\longrightarrow} G \stackrel{\beta}{\longrightarrow} G/N \longrightarrow 1$ ist exakt für jeden Normalteiler $N$ von $G$.
		\item Für G abelsch ($\{e\} =: 0$): \\
		$0 \longrightarrow H \stackrel{\alpha}{\longrightarrow} G \stackrel{\beta}{\longrightarrow} G/H \longrightarrow 0$ ist exakt für jede Untergruppe $H$ von $G$.
	\end{itemize}
\end{ex}



\section{Produkte von Gruppen}

\begin{defn}
	Es sei $(G_i)_{i \in I}$ eine Familie von Gruppen. Das \underline{äußere direkte Produkt} der Familie ist das kartesische Produkt $\prod_{i \in I}G_i$ mit der Verknüpfung $(a_i)_{i \in I} (b_i)_{i \in I} = (a_ib_i)_{i \in I}$. Neutrales Element: $(e_i)_{i \in I}$ mit $e_i \in G_i$ neutral.
\end{defn}

\begin{notation}
	$G_1 \times G_2 \times \dotsb \times G_n$ für endliche Produkte. \\
	$G_1 \oplus G_2 \oplus \dotsb \oplus G_n$ für endliche Produkte, $G_i$ abelsch [additiv].
\end{notation}

\begin{lemma}
	Für jedes $i_0 \in I$ ist die Teilmenge
	$$ \overline{G_{i_0}} = \{(b_i)_{i \in I} \in \prod_{i \in I}G_i | b_i = e_i \text{ für } i \neq i_0\} $$
	ein Normalteiler von $\prod_{i \in I}G_i$, isomorph zu $G_{i_0}$.
\end{lemma}

\begin{proof}
	$\overline{G_{i_0}}$ ist der Kern des Gruppenhomomorphismus
	$$p_{i_0}: \prod_{i \in I}G_i \rightarrow \prod_{i \in I-\{i_0\}}G_i, (b_i)_{i \in I} \mapsto (b_i)_{i \in I-\{i_0\}}.$$
	Ferner ist
	$$j_{i_0}: G_{i_0} \rightarrow \prod_{i \in I}G_i, a \mapsto (b_i)_{i \in I} \text{ mit } b_{i_0}=a \text{ und } b_i = e_i \text{ für } i \neq i_0.$$
	Das Bild ist $\overline{G_{i_0}}$ und $j_{i_0}$ ist injektiv, weil
	$$(b_i)_{i \in I} = (e_i)_{i \in I} \Leftrightarrow b_i = e_i \text{ für alle } i \in I.$$
\end{proof}

\begin{nb}
	\begin{itemize}
		\item $1 \longrightarrow G_{i_0} \stackrel{j_{i_0}}{\longrightarrow} \prod_{i \in I}G_i \stackrel{p_{i_0}}{\longrightarrow} \prod_{i \in I-\{i_0\}}G_i \longrightarrow 1$ ist kurze exakte Folge.
		\item Der Beweis liefert $\overline{G_{i_0}} \cap \langle \bigcup_{i \in I-\{i_0\}}G_i \rangle = \{e\}$.
	\end{itemize}
\end{nb}

\begin{defn}
    Sei $G$ eine Gruppe und $(N_i)_{i\in I}$ eine Familie von normalen Untergruppen, so dass gilt:

    \begin{enumerate}[label=(\roman*)]
        \item $G=<\cup_{i\in I} N_i>$
        \item $N_{i_0} \cap <\cup_{i\in I \setminus \{i_0\}} N_i> = \{e\}$ für jedes $i_0\in I$.
    \end{enumerate}

    Dann ist $G$ das \underline{innere Produkt} von $(N_i)_{i\in I}$.
\end{defn}

\begin{lemma}
    Es sei $G$ das innere Produkt von $(N_i)_{i\in I}$. Dann gilt $ab=ba$ für $a\in N_i, b\in N_j$ mit $i\neq j$.
\end{lemma}

\begin{proof}
    Man rechnet: $$ab\inv{ba}=ab\Inv{a}\Inv{b}=(ab\Inv{a})\Inv{b}=a(b\Inv{a}\Inv{b}) \in N_i\cap N_j$$ Somit folgt $ab\inv{ba}=e$, also $ab=ba$.
\end{proof}

Wir werden uns demnächst nur mit endlichen Produkten beschäftigen.

\begin{lemma}
    Sei $G$ Gruppe, $N_1,\dots,N_r$ normale Untergruppen von $G$. Dann ist $G$ genau dann das innere Produkt von $N_1,\dots,N_r$, wenn gilt:

    \begin{enumerate}[label=(\roman*)']
        \item $G=N_1\dots N_r$
        \item Die Darstellung $a=n_1\dots n_r$ mit $n_j\in N_j$ ist für jedes $a\in G$ eindeutig bestimmt.
    \end{enumerate}
\end{lemma}

\begin{proof}
    (i) äquivalent (i)': folgt aus $$<\bigcup_{i=1}^r N_i> = \{a_1^{\xi_1}\dots a_k^{\xi_k}: a_j\in N_1\cup\dots \cup N_r, \xi_j = \pm 1\}$$ und der Tatsache, dass die $N_j$ normale Untergruppen sind die für paarweise verschiedene $j\neq j'$ kommutieren.

    (ii)' impliziert (ii): oBdA gelte $i_0=1$. Wir möchten zeigen, dass $$N_1\cap <N_2\cup\dots \cup N_r> = N_1 \cap (N_2\dots N_r)= \{e\}$$ Sei $x\in N_1\cap <N_2\dots N_r>$. Dann gilt $x=n_1\in N_1$ und $x=n_2\dots n_r$ mit $n_i\in N_i$ für $i\geq 2$. Also $x=n_1e\dots e=en_2\dots n_r$, somit $n_i=e$ für alle $i$, also $x=e$.

    (ii) impliziert (ii)': Aus $n_1\dots n_r=n_1'\dots n_r'$ mit $n_i,n_i'\in N_i$ folgt $$\inv{n_1'}n_1=(n_2'\dots n_r')\inv{n_2\dots n_r} = \{e\}$$ Die letzte Gleichheit gilt nach (ii). Also folgt $n_i=n_i'$. Vertauschen der Reihenfolge liefert dies für $i\neq 1$.
\end{proof}

\begin{thm}
    Das innere Produkt $G$ von normalen Untergruppen $N_1,\dots,N_r$ ist zum äußeren Produkt $N_1\times\dots \times N_r$ kanonisch isomorph. Folgerung: Wir brauchen nicht zwischen den $\overline{G_i}=N_i$ und den $G_i$ zu unterscheiden.
\end{thm}

\begin{proof}
    Wir definieren $$N_1\times\dots\times N_r \overset{\pi}{\longrightarrow} G,\quad (n_1,\dots,n_r)\mapsto n_1\dots n_r$$ Da das Bild von $\pi$ die Untergruppe $N_1\dots N_r \overset{(i)'}{=} G$ ist, ist $\pi$ surjektiv. Aus (ii)' folgt, dass $\pi$ auch injektiv ist. $\pi$ ist auch Gruppenhomomorphismus: 
    
    \begin{align*}
        \pi((m_1,\dots,m_r)\cdot (n_1,\dots,n_r)) \\
        =\pi((m_1n_1,\dots,m_rn_r))=m_1n_1\dots m_rn_r=m_1\dots m_rn_1\dots n_r \\
        \pi(\bar m)\pi(\bar n)
    \end{align*}
    
    Dabei gilt die vorletzte Gleichheit, weil wir Elemente aus verschiedenen Untergruppen vertauschen dürfen.
\end{proof}

\begin{defn}
    Für $N,H$ Gruppen ist eine Gruppenerweiteruung von $N$ durch eine kurze, exakte Folge von Gruppen
    
    \[ \begin{tikzcd}
        1 \arrow[r] & N \arrow[r] & G \arrow[r] & H \arrow[r] & 1
    \end{tikzcd} \]
    
    gegeben, wobei $i$ injektiv, $p$ surjektiv, $\text{im } i = \ker p$ ($\nicefrac{G}{H} \cong H$).

\end{defn}

\begin{ex}
    $G=N\times H$ Diese ist i.A. nicht die einzige Gruppenerweiterung.
\end{ex}

\begin{defn}
    Es seien $G$ eine Gruppe, $N\subseteq G$ Normalteiler von $G$ und $H\subseteq G$ beliebige Untergruppe, so dass gilt: $$G=NH \quad \text{und} \quad N\cap H = \{e\}$$ Dann ist $G$ das \underline{semidirekte Produkt} von $N$ und $H$, Notation: $G=N\rtimes H$.
\end{defn}

\begin{nb}
    Jedes $a\in G$ hat eine eindeutige Darstellung $a=nh$ mit $n\in N, h\in H$. Dies liefert eine Bijektion $$N\times H \to G=N\rtimes H, (n,h)\mapsto nh$$
\end{nb}

\begin{thm}
    Für jedes $h\in H$ ist die Konjugationsabbildung $$\gamma_h: N\to N, \quad n\mapsto hn\Inv{h}$$ ein Automorphismus von $N$. Dies ergibt den Homomorphismus $$H\overset{\gamma}{\longrightarrow} \text{Aut}(N), h\mapsto \gamma_h$$
\end{thm}

\begin{proof}
    $\gamma$ ist Homomorphismus:

    \begin{gather*}
        \gamma_{h_1h_2}(n)=(h_1h_2)n\inv{h_1h_2}=h_1h_2n\Inv{h_2}\Inv{h_1} \\
        = h_1\gamma_{h_2}(n)\Inv{h_1}=(\gamma_{h_1}\circ\gamma_{h_2})(n)
    \end{gather*}

    für $n\in N$.
\end{proof}

\begin{nb}
    \begin{enumerate}[label=(\arabic*)]
        \item Das semidirekte Produkt von $N$ und $H$ wird von $\gamma: H\to \text{Aut}(N)$ eindeutig festgelegt. Es gilt nämlich: $$(n_1h_1)(n_2h_2)=n_1(h_1n_2\Inv{h_1})h_1h_2=n_1\gamma_{h_1}(n_2)h_1h_2$$ wobei $$n_1\gamma_{h_1}(n_2)\in N, h_1h_2\in H$$.
        \item Umgekehrt definiert $\gamma: H\to \text{Aut}(N)$ immer ein semidirektes Produkt $N\rtimes H$:

            \begin{gather*}
                G=(N\times H, \cdot_\gamma) \\
                (n_1,h_1)\cdot_\gamma (n_2,h_2)=(n_1\gamma_{h_1}(n_2), h_1h_2)
            \end{gather*}

            ist Gruppe mit neutralem Element $(e_N, e_H)$ und inversen Elementen $$\Inv{(n,h)}=(\gamma_{\Inv{h}}(\Inv{n}),\Inv{h})$$

            Definiere:

            \begin{gather*}
                N^*\coloneqq = \{(n, e_H): n\in N\} \cong N, \\
                H^*\coloneqq = \{(e_N, h): h\in H\} \cong H
            \end{gather*}

            $N^*,H^*$ sind Untergruppen von $G$, isomorph zu $N$ bzw. $H$. $$\pi: G\longrightarrow H, \quad (n,h)\mapsto h$$ ist ein Homomorphismus mit $\ker \pi = N^*$, also ist $N^*$ normale Untergruppe.

            Das $N^*\cap H^* = \{e\}$ ist klar. $$(n,h)=(n,e_H)\cdot_\gamma (e_N, h)$$ liefert $G=N^*H^*$.
    \end{enumerate}
\end{nb}

\begin{defn}
    Eine kurze exakte Folge 

    \[ \begin{tikzcd}
        1 \arrow[r] & N \arrow[r, "i"] & G \arrow[r, "p"] & H \arrow[l, bend left, dashed, "s"] \arrow[r] & 1
    \end{tikzcd} \]

    spaltet, falls ein Homomorphismus $s: H\to G$ existiert, mit $p\circ s = \text{id}_h$. $p$ heißt ein \underline{Schnitt} von $p$.
\end{defn}

Für $G=N\rtimes H$ Gruppenerweiterung:

\[ \begin{tikzcd}
        1 \arrow[r] & N \arrow[r,"i"] & G \arrow[r,"p"] & H \arrow[r] & 1
\end{tikzcd} \]

wobei $i: N\to G, n\mapsto (n,e_H), p:G\to H, (n,h)\mapsto h$ ist $j:H\to G, h\mapsto (e_N,h)$ Schnitt. $H$ ist Untergruppe. Man rechnet $(p\circ j)(h)=p(e_N,h)=h$

\begin{thm}
    Es sei

    \[ \begin{tikzcd}
        1 \arrow[r] & N \arrow[r] & G \arrow[r] & H \arrow[r] & 1
    \end{tikzcd} \]

    eine Gruppenerweiterung, die mit einem Schnitt $s:H\to G$ spaltet. Dann ist $G$ das semidirekte Produkt von $N$ und $H$, definiert durch $$\gamma: H\to \text{Aut}(N), \quad h\mapsto \gamma_h$$
\end{thm}

\begin{proof}
    Wir setzen $\rho: N\rtimes H \to G, (n,h)\mapsto i(n)s(h)$. Zeige, dass $\rho$ Homomorphismus ist:

    \begin{gather*}
        \rho((n_1,h_1)\cdot (n_2,h_2)) = \rho(n_1\gamma_{h_1}(n_2), h_1h_2) \\
        = i(n_1)i(\gamma_{h_1}(n_2))s(h_1)s(h_2) = i(n_1)s(h_1)i(n_2)\Inv{s(h_1)}s(h_1)s(h_2) \\
        = \rho(n_1,h_1)\rho(n_2,h_2)
    \end{gather*}
\end{proof}
 
\end{document}
