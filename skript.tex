\documentclass[12pt]{scrartcl}%{article} % Beginn der LaTeX-Datei
\usepackage{amsmath,amssymb,amsthm}  % erleichtert Mathe 
\addtolength{\jot}{0.2em}
\usepackage{enumitem}% schicke Nummerierung
\newcommand{\sbt}{\,\begin{picture}(-1,1)(-1,-3)\circle*{3}\end{picture}\ }
\renewcommand{\labelitemi}{\sbt}

\usepackage{graphicx} % für Grafik-Einbindung
\usepackage{float} 
%\usepackage[dvips]{hyperref}
\usepackage[arrow, matrix]{xy} % für kommutative Diagramme

\usepackage[ngerman]{babel}
\usepackage[autostyle=true,german=quotes]{csquotes}
%\usepackage[T1]{fontenc}
%\usepackage{lmodern}
% Einstellungen, wenn man deutsch schreiben will, z.B. Trennregeln
%\usepackage[applemac]{inputenc}
\usepackage[utf8]{inputenc}  % für Unix-Systeme


% environments
\newtheorem{thm}{Theorem}
\newtheorem{lemma}{Lemma}
\newtheorem{kor}{Korollar}
% definition-like stuff
\theoremstyle{definition}
\newtheorem*{defn}{Definition}
\newtheorem{ex}{Beispiel}
% remark-like stuff
\theoremstyle{remark}
\newtheorem*{notation}{Notation}
\newtheorem*{nb}{Bemerkung}

% commands
\newcommand{\powerset}{\mathcal{P}}
\newcommand{\sym}{\text{Sym}}
\newcommand{\gl}{\text{GL}}
\newcommand{\abb}{\text{Abb}}
\newcommand{\inv}[1]{\left(#1\right)^{-1}}
\newcommand{\Inv}[1]{#1^{-1}}

\begin{document}

\author{Sebastian Bechtel, Isburg Knof, Theresa Tran}
\title{Einführung in die Algebra}
\date{15. April 2015}

\maketitle

\section{Gruppen}

\begin{defn}
    Eine (innere) \underline{Verknüpfung} auf einer Menge $M\neq \emptyset$ ist eine Abbildung $M\times M\to M, (a,b)\mapsto a\cdot b$.
\end{defn}

\begin{defn}
    Eine \underline{Gruppe} ist eine Menge $G\neq \emptyset$ zusammen mit einer Verknüpfung $\cdot$, sodass Assoziativität (A), Existenz eines neutralen Elements (N) und Existenz inverser Elemente (I) erfüllt sind. $G$ ist \underline{abelsch}, falls Kommutativität (K) gilt.
\end{defn}

\begin{ex}
    \begin{enumerate}
        \item $\mathbb{Z}, \mathbb{Q}, \mathbb{R}, \mathbb{C}$ sind abelsche Gruppen mit $+$ als Verknüpfung.
        \item $\mathbb{Q}^*=\mathbb{Q}\setminus \{0\}, \mathbb{R}^*, \mathbb{C}^*$ mit Multiplikation sind abelsche Gruppen.
        \item Für eine Menge $M$ ist $\sym(M)$ ist eine Gruppe, aber nicht abelsch.
    \end{enumerate}
\end{ex}

\begin{lemma}
    \begin{enumerate}[label=\alph*)]
        \item Das neutrale Element ist eindeutig. :)
        \item Inverse Elemente sind eindeutig.
    \end{enumerate}
\end{lemma}

\begin{proof}
    \begin{enumerate}[label=\alph*)]
        \item Seien $e,f$ neutrale Elemente, dann gilt $e=ef=f$.
        \item Sei $a\in G$ und $b,b'\in G$ inverse Elemente. Dann gilt $b'=b'e=b'(ab)=(b'a)b=eb=b$. 
    \end{enumerate}
\end{proof}

\begin{notation}
    multiplikativ: $a\cdot b$ oder $ab$, neutrales Element $e$ oder $1$, inverses Element von $a\in G$ ist $\Inv a$.
\end{notation}

\begin{lemma}
    Es sei $\mathcal{G}=(G,\cdot)$ eine Menge mit assoziativer Verknüpfung, einem linksneutralen Element und linksinversen Elementen, dann ist $\mathcal{G}$ eine Gruppe.
\end{lemma}

\begin{proof}
    Sei $a\in G$ und $b\in G$ mit $ba=e$. Nach (I') gibt es $c\in G$ mit $cb=e$. Also $ab=eab=cbab=ceb=cb=e$.

    Sei nun $a\in G$, es gilt $ae=a(\Inv aa)=ea=e$.
\end{proof}

\begin{lemma}
    \begin{enumerate}
        \item $\inv{\Inv{a}}=a$, $\inv{ab}=\Inv b\Inv a$
        \item $ab=ac$ impliziert $b=c$ für alle $a,b,c\in G$.
        \item für $a,b\in G$ gibt es genau ein $x\in G$, sodass $ax=b$.
    \end{enumerate}
\end{lemma}

\begin{proof}
    \begin{enumerate}
        \item $\inv{\Inv a}=a$ klar. Für $a,b,c\in G$: $(\Inv b\Inv a)ab=\Inv b(\Inv aa)b=\Inv beb=\Inv bb=e$ (andere Richtung analog)
        \item $ab=ac$ impliziert $\Inv a(ab)=\Inv a(ac)$ impliziert $b=c$
        \item Setze $x=\Inv ab$, dann erhält man $ax=a(\Inv ab)=(a\Inv a)b=eb=b$
    \end{enumerate}
\end{proof}

\begin{defn}
    Sei $a\in G$, $(G,\cdot)$ Gruppe. Für $n\in \mathbb{Z}$ definiere:
    
    $$a^0:=e, \quad a^n:=a^{n-1}a \quad \text{ für } n\geq 1$$
    $$a^n:=\left(\Inv a\right)^{-n} \quad \text{ für } n < 0$$
\end{defn}

\begin{lemma}
    Für $a\in G$ gilt: $a^n a^m=a^{n+m}=a^m a^n$, $\left(a^m \right)^n = a^{n\cdot m}$, $ab=ba$ impliziert $\left(ab \right)^n = a^n b^n$
\end{lemma}

\begin{ex}
    \begin{enumerate}
        \item $K$ Körper, dann ist $\gl_n(K)$ ein Gruppe bzgl. Matrixmultiplikation.
        \item $M\neq \emptyset$ Menge, $(G, \cdot)$ Gruppe, definiere $\abb(M,G):=G^M$. Für $f,g\in \abb(M,G)$ ist $f\cdot g$ gegeben durch $(f\cdot g)(m)=f(m)\cdot g(m)$ für $m\in M$. Dann ist $(\abb(M,G), \cdot)$ eine Gruppe.
    \end{enumerate}
\end{ex}



\section{Untergruppen}

\begin{defn}
    Sei $(G, \cdot)$ Gruppe. Eine Teilmenge $H\subset G$ heißt Untergruppe von $G$, falls $(H, \cdot)$ eine Gruppe ist.

    Äquivalent dazu:

    \begin{enumerate}[label=(\roman*)]
        \item Für $a,b\in H$ gilt $ab\in H$ (Abgeschlossenheit)
        \item $e\in H$
        \item für $a\in H$ ist $\Inv a \in H$
    \end{enumerate}
\end{defn}

\begin{thm}
    Sei $(G, \cdot)$ Gruppe und $H\subset G$ nicht-leer. Dann gilt: $H$ induziert Untergruppe von $(G, \cdot)$ gdw. $a\Inv b\in H$ für $a,b\in H$.
\end{thm}

\begin{proof}
    "$\Rightarrow$" \checkmark

    "$\Leftarrow$" 

    \begin{itemize} 
        \item $a=b$ impliziert $e\in H$
        \item $e,a\in H$ impliziert $e\Inv a\in H$ impliziert $\Inv a \in H$
        \item $a,\Inv b\in H$ impliziert $a\inv{\Inv b} \in H$ impliziert $ab\in H$
    \end{itemize}
\end{proof}

\begin{ex}
    \begin{enumerate}[label=(\alph*)]
        \item $\{e\}, G$ induziert Untergruppe für alle Gruppen $(G,\cdot)$.
        \item $K$ Körper. $\text{SL}_n(K)=\{A\in M_n(K): \det(A)=1\}$ induziert Untergruppe von $\gl_n(K)$, die spezielle lineare Gruppe.
    \end{enumerate}
\end{ex}

\begin{defn}
    Eine Untergruppe heißt \underline{echt}, falls sie nicht trivial ist.
\end{defn}

\begin{lemma}
	Es sei $(H_{j})_{j \in J}$ eine Familie von Untergruppen $H_{j} \subset G$. Dann ist $\bigcap_{j 		\in J}H_{j}$ eine Untergruppe von G.
\end{lemma}

\begin{proof}
	Übung
\end{proof}

\begin{defn}
	Es sei M eine Teilmenge von G. Die \underline{von M erzeugte Untergruppe} ist der Durchschnitt 	aller Untergruppen, die M enthalten.
\end{defn}

\begin{notation}
	$\langle M \rangle =\bigcup_{M \subset H \subset G}H$, wobei H Untergruppe
\end{notation}

\begin{nb}
	\begin{enumerate}[label=(\alph*)]
		\item $\langle \emptyset \rangle = \lbrace e \rbrace$
		\item Für $M \neq \emptyset$ gilt: $\langle M \rangle = \lbrace m_{1}^{\varepsilon_{1}} \cdot ... \cdot m_{n}^{\varepsilon_{n}} : m_{1},...,m_{n} \in M, \varepsilon_{1},...,\varepsilon_{n} \in \lbrace -1,+1\rbrace, n \geq 0 \rbrace$
		\item Für $M = \lbrace g \rbrace$ gilt: $\langle g \rangle = \lbrace g^{n} : n \in \mathbb{Z} \rbrace$. Von g erzeugte zyklische Untergruppe von G.
	\end{enumerate}
\end{nb}

\begin{defn}
	G heißt \underline{zyklisch}, falls $G = \langle g \rangle$ für ein $g \in G$ gilt. \newline
	Ist $G = \langle M \rangle$ mit M endlich, so heißt G \underline{endlich erzeugt}.
\end{defn}

\begin{defn}
	\begin{enumerate}[label=(\roman*)]
		\item Die \underline{Ordnung einer Gruppe} $G$ ist $ord(G)=\vert G \vert$.
		\item Die \underline{Ordnung eines Elements} $g \in G$ ist $ord(g)=ord(\langle g \rangle)$.
		\item Ist $ord(g)$ endlich, dann hat g \underline{endliche Ordnung}.
	\end{enumerate}
\end{defn}

\begin{notation}
    $(n,s)$ bezeichnet den größten gemeinsamen Teiler.
\end{notation}

\begin{thm}
	Sei $G$ Gruppe, $g \in G$
	\begin{enumerate}
		\item g hat endliche Ordnung $\Longleftrightarrow$ alle Potenzen von g sind verschieden
		\item g hat endliche Ordnung $\Longleftrightarrow$ $\exists m>0: g^{m}=e$ \newline Dann gilt:
			\begin{enumerate}[label=(\alph*)]
				\item $n := ord(g) = min \lbrace m>0 : g^{m}=e \rbrace$
				\item $g^{m}=e \Longleftrightarrow m=nk$ mit $k \in \mathbb{Z}$
				\item $\langle g \rangle = \lbrace e, g^{1},...,g^{n-1}\rbrace$
			\end{enumerate}
		\item $ord(g^{s}) = \frac{n}{(n,s)}$ für $n = ord(g)$ endlich
	\end{enumerate}
\end{thm}

\begin{proof}
	\begin{enumerate}
		\item Wir nehmen an: Für $i,j \in \mathbb{Z}$, oBdA $j>i$ gilt $g^{i}=g^{j}$. \newline Dann gilt $g^{j-i}=g^{j}(g^{i})^{-1}=e$. \newline Es sei dann n die kleinste positive Zahl, die $g^{n}=e$ erfüllt. Sei $m \in \mathbb{Z}$ beliebig. Der Divisionsalgorithmus liefert: $m=kn+r$ für $0 \leq r < n$ und $k,r \in \mathbb{Z}$. Dann gilt: \newline $g^{m}=g^{kn+r}=g^{kn}g^{r}=(g^{n})^{k}g^{r}=eg^{r}=g^{r}$. \newline Daraus folgt $\langle g \rangle = \lbrace g^{m} : m \in \mathbb{Z}\rbrace = \lbrace g^{r} : r=0,...,n-1\rbrace$. Besonders gilt $ord(g)=n$ ist endlich. \newline $\lceil$ Dies zeigt $\Rightarrow$, $\Leftarrow$ klar, dann ist $\langle g \rangle = \lbrace g^{m} : m \in \mathbb{Z}\rbrace$ unendlich $\rfloor$
		\item Alle $g^{r}$ mit $0 \leq r \leq n-1$ sind verschieden, da: \newline $g^{i}=g^{j} \Rightarrow g^{j-i}=e \Rightarrow j-i = kn$ mit $k \in \mathbb{Z} \Rightarrow i \equiv j (modn) \Rightarrow i=j$ falls $0<i,j<n-1$. \newline Dies liefert $g^{r}$ mit $0 \leq r \leq n-1$ sind paarweise verschieden und es gilt: \newline $ord(g)=n$. $\lceil a$ und $c\rfloor$ \newline Aus dem Divisionsalgorithmus folgt b: $g^{m}=e \Leftrightarrow e=g^{kn+r}=g^{r}$ mit $m = kn+r, 0 \leq r < n \Leftrightarrow r=0$. Also $m=kn$ mit $k \in \mathbb{Z}$.
		\item Es sei $m=ord(g^{s}), n =ord(g)$. Aus $(g^{s})^{m}=e$ folgt (siehe 2), dass $n$ ein Teiler von $sm$ ist. Dies liefert: $\frac{n}{(s,n)} \vert \frac{s}{(s,n)}m$. Somit $\frac{n}{(s,n)} \vert m$. \newline Nun möchten wir noch zeigen: $m \vert \frac{n}{(s,n)}$. $(g^{s})^{\frac{n}{(s,n)}} = (g^{n})^{\frac{s}{(s,n)}}=e^{\frac{s}{(s,n)}}=e$. Daraus folgt $m \vert \frac{n}{(s,n)}$ (wegen 2). \newline Also gilt $m = \frac{n}{(s,n)}$. 
	\end{enumerate}
\end{proof}

\begin{lemma}
	Wir können alle Untergruppen einer zyklischen Gruppe beschreiben mit $G = \langle g \rangle, H \subset G$, es sei $h \in H, h \neq e$. Dann gilt: $h = g^{k}$.
\end{lemma}

\begin{proof}
	Wir setzen: $m = min \lbrace k>0 : g^{k} \in H \rbrace$. \newline $\lceil$ Existiert: $G= \langle g \rangle = \langle g^{-1} \rangle, h = g^{k}, k<0$, dann ersetzen wir $h$ durch $h^{-1} \rfloor$ \newline Wir wollen zeigen: $\langle g^{m} \rangle = H$ 
	\begin{enumerate}
		\item $\langle g^{m} \rangle \subset H$ gilt wegen $g^{m} \in H$
		\item Es sei $j \in  \mathbb{Z}$ mit $g^{j} \in H$. Divisionsalgorithmus liefert $j=lm+r$ mit $0 \leq r < m$: $g^{j} \in H \Rightarrow g^{r}=g^{-lm}g^{lm+r}=(g^{m})^{-l}g^{j}$. Also $g^{r} \in H$. Aus der Minimalität von M folgt $r=0$. Dies liefert $g^{j}=(g^{m})^{l} \in \langle g^{m} \rangle$ und somit gilt: $H \subset \langle g^{m} \rangle$ und die zwei Untergruppen stimmen überein.
	\end{enumerate}
\end{proof}

Ähnlich kann man zeigen:

\begin{thm}
	Alle Untergruppen einer zyklischen Gruppe sind zyklisch. Ist $ord(G)=n$ endlich und $m$ ein Teiler von $n$, so ist $H = \langle g^{\frac{n}{m}}\rangle$ die einzige Untergruppe der Ordnung $m$.
\end{thm}

\begin{defn}
	Sei $H$ eine Untergruppe der Gruppe $G$. Dann kann man die folgende Äquivalenzrelation definieren: \newline $(x,y) \in G^{2}: x \sim_{H} y \Leftrightarrow x = yh$ für $h \in H$ \newline $\lceil$Äquivalenzrelation wegen Gruppenaxiomen für $H\rfloor$
\end{defn}

\begin{defn}
	Die Äquivalenzklassen bzgl. $\sim_{H}$ heißen \underline{Linksnebenklassen}.
\end{defn}

\begin{notation}
	Für $a \in G, aH = {ah : h \in H}$
\end{notation}

\begin{nb}
	Es gelten folgende Eigenschaften: 
	\begin{itemize}
		\item Die Abbildung $H \rightarrow aH, h \mapsto ah$ ist eine Bijektion. Besonders gilt: $\vert aH \vert = \vert H \vert$ für alle $a \in G$. \newline $\lceil$Die Abbildung ist bijektiv, da sie umkehrbar ist: $aH \rightarrow H, b \mapsto a^{-1}b$ ist die Umkehrfunktion$\rfloor$
		\item $aH \neq bH \Rightarrow aH \cap bH = \emptyset$, d.h. sie sind disjunkt. \newline $\lceil x \in aH \neq bH \Rightarrow x = ah_{1} = bh_{2}$ für $h_{1},h_{2} \in H \Rightarrow a=bh_{2}h_{1}^{-1} \in bH \Rightarrow ah= b(h_{2}h_{1}^{-1}h) \in bH$ für alle $h \in H \Rightarrow aH \subset bH$. Ähnlich gilt $bH \subset aH$. Daraus folgt $aH=bH.\rfloor$
	\end{itemize}
\end{nb}

\begin{defn}
	$G/H = \lbrace aH : a \in G \rbrace$ ist die \underline{Menge der Linksnebenklassen}. \newline Der Index von $H$ ist die Mächtigkeit von $G/H$, d.h. \underline{Index} [$G:H$]$:=\vert G/H \vert$
\end{defn}

\begin{nb}
	\begin{itemize}
	 	\item $\vert G \vert = [G:H]\vert H\vert$
	 	\item Analog ist $a \sim_{H} b$ mit $a,b \in G \Leftrightarrow a=hb$ für ein $h \in H$ ("rechtsäquivalent bzgl. H") eine Äquivalenzrelation. \newline \underline{Rechtsnebenklassen}: $Ha=\lbrace ha : h \in \mathbb{Z} \rbrace$ mit $a \in G$ \newline Bijektion: Für $a \in G$ $aH \rightarrow Ha, x \mapsto a^{-1}xa$
	\end{itemize}
\end{nb}

\begin{defn}
	$H \backslash G$ ist die \underline{Menge der Rechtsnebenklassen}. Dann gilt: $\vert H \backslash G \vert = \vert G/H \vert$ \newline $\lceil$Bijektion: $H \backslash G \rightarrow G/H, Hb \mapsto b^{-1}H \rfloor$
\end{defn}



% Hier fehlt die Vorlesung vom 28.April. Viel Spaß beim texen, Philipp :D

\newpage
\thispagestyle{empty}
Hier fehlt die Vorlesung vom 28.April. Viel Spaß beim texen, Philipp :D

\newpage





\begin{defn}
	A, B, C Gruppen, $ f: G \rightarrow H$ Gruppenhomomorphismus. \\
	Das \underline{Bild von $f$} ist: $im(f) = \{f(g) | g \in G\} \subseteq H$. \\
	Der \underline{Kern von $f$} ist: $ker(f) = \{g \in G | f(g) = e_H\} \subseteq G$.
\end{defn}

\begin{lemma}
	Für jeden Gruppenhomomorphismus $ f: G \rightarrow H$ gilt:
	\begin{enumerate}[label=(\roman*)]
		\item $im(f)$ ist eine Untergruppe von H.
		\item $ker(f)$ ist eine normale Untergruppe von G.
		\item $f$ injektiv (bzw. surjektiv) $ \Leftrightarrow ker(f) = {e_G}$ (bzw. $im(f) = H$).
	\end{enumerate}
\end{lemma}

\begin{proof}
	\begin{enumerate}[label=(\roman*)]
		\item $im(f)$ ist wegen der Definition von Gruppenhomomorphismen unter $\cdot_H$ abgeschlossen, enthält $e_H = f(e_G)$ (Eigenschaft (i)) und die Inversen aller seiner Elemente (Eigenschaft (ii)).
		\item Für alle $a, b \in ker(f)$ gilt:
			$$ f(ab^{-1}) = f(a)f(b^{-1}) = f(a)(f(b))^{-1} = ee^{-1} = e.$$
			Somit ist $ker(f)$ eine Untergruppe. \\
			Für jedes $ g \in ker(f)$ und $x \in G$ gilt:
			$$ f(x^{-1}gx) = f(x^{-1})f(g)f(x) = f(x^{-1}) e f(x) = f(x^{-1})f(x) = f(x^{-1}x) = f(e) = e.$$
			Also ist $x^{-1}gx \in ker(f) $ und somit ist $ker(f)$ ein Normalteiler.
		\item Subjektivität: offentsichtlich. Injektivität: vgl. Übung 2, Aufgabe 1.
	\end{enumerate}	
\end{proof}

\begin{nb}
	\begin{itemize}
		\item Ein bijektiver Homomorphismus wird \underline{Isomorphismus} genannt. Die Umkehrfunktion ist dann wieder ein Isomorphismus.
		\item Ein Gruppenhomomorphismus $G \rightarrow G$ heißt \underline{Endomorphismus von G}.
		\item Ein Gruppenisomorphismus $G \rightarrow G$ heißt \underline{Endomorphismus von G}.
	\end{itemize}
\end{nb}

\begin{defn}
	$G/N$ heißt \underline{Faktorgruppe von $N$ in $G$}, wenn $N$ normale Untergruppe von $G$ ist. \\
	Gruppenstruktur: $aN \cdot bN = abN$, $eN = N$ neutrales Element \\
	Die \underline{natürliche Projektion} $\pi : G \rightarrow G/N, a \mapsto aN$ ist ein surjektiver Gruppenhomomorphismus mit $ker(\pi) = N, im(\pi) = G/N$.
\end{defn}

\begin{thm}[Homomorphiesatz]
	Es sei $f: G \rightarrow G'$ ein Gruppenhomomorphismus und N ein Normalteiler von G. Ist N im $ker(f)$ enthalten, so gibt es genau einen Gruppenhomomorphismus $\bar{f}: G/N \rightarrow G'$, so dass das folgende Diagramm kommutiert:
	\centering
	\begin{xy}
 		\xymatrix{
      			G \ar[rr]^{f = \bar{f} \circ \pi} \ar[rd]_{\pi}	&					& G'	\\
      											& G/N \ar[ur]_{\bar{f}}	&
  		}
	\end{xy}
\end{thm}

\begin{nb}
	Es gilt $ker(\bar{f}) = \pi(ker(f))$ und $im(\bar{f}) = im(f)$. 
\end{nb}

\begin{kor}
	Wenn $f: G \rightarrow G'$ ein surjektiver Homomorphismus ist, dann ist $\bar{f}: G/ker(f) \rightarrow G'$ ein Isomorphismus.
\end{kor}

\begin{proof}[Beweis des Satzes]
	Aus dem kommutativen Diagramm folgt, dass $\bar{f}(aN) = f(a)$ sein sollte für jedes $aN \in G/N$. \\
	Wir zeigen zuerst, dass $\bar{f}$ wohldefiniert ist. Es seien $a, b \in G$ mit $aN = bN$. Dann gilt $a \sim_{N} b$ und also $b^{-1}a \in N$. Daraus folgt
	$$f(a) = f(ea) = f(b(b^{-1}a)) = f(b) f(b^{-1}a) \stackrel{b^{-1}a \in N \subseteq ker(f)}{=} f(b)e = f(b).$$
	Die Abbildung $\bar{f}: G/N \rightarrow G'$ existiert somit und ist offensichtlich ein Gruppenhomomorphismus, denn
	$$\bar{f}(abN) = f(ab) = f(a)f(b) = \bar{f}(aN)\bar{f}(bN)$$ for all $a, b \in G$.
\end{proof}

\begin{proof}[Beweis der Bemerkung]
		\begin{itemize}
		\item $aN \in ker(\bar{f}) \stackrel{Def. ~ von ~ \bar{f}}{\Leftrightarrow} a \in ker(f)\stackrel{Def. ~ von ~ \pi}{\Leftrightarrow} aN \in \pi(ker(f))$
		\item $im(f) = im(\bar{f} \circ \pi) = \bar{f}(im(\pi)) \stackrel{\pi ~ surj.}{=} \bar{f}(G/N) = im(\bar{f})$
	\end{itemize}
\end{proof}

\begin{thm}[1. Isomorphiesatz]
	
\end{thm}

\begin{proof}
	
\end{proof}

\begin{thm}[2. Isomorphiesatz]
	
\end{thm}

\begin{proof}
	
\end{proof}

\begin{notation}
	
\end{notation}

\begin{ex}
	
\end{ex}



\section{Produkte von Gruppen}

\begin{defn}
	
\end{defn}

\begin{notation}
	
\end{notation}

\begin{lemma}
	
\end{lemma}

\begin{proof}
	
\end{proof}

\begin{nb}
	
\end{nb}

\end{document}