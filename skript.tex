\documentclass[12pt]{scrartcl}%{article} % Beginn der LaTeX-Datei
\usepackage{amsmath,amssymb,amsthm}  % erleichtert Mathe 
\addtolength{\jot}{0.2em}
\usepackage{enumitem}% schicke Nummerierung
\newcommand{\sbt}{\,\begin{picture}(-1,1)(-1,-3)\circle*{3}\end{picture}\ }
\renewcommand{\labelitemi}{\sbt}

\usepackage{graphicx} % für Grafik-Einbindung
\usepackage{float} 
%\usepackage[dvips]{hyperref}

\usepackage[ngerman]{babel}
\usepackage[autostyle=true,german=quotes]{csquotes}
%\usepackage[T1]{fontenc}
%\usepackage{lmodern}
% Einstellungen, wenn man deutsch schreiben will, z.B. Trennregeln
\usepackage[utf8]{inputenc}  % für Unix-Systeme

% environments
\newtheorem{thm}{Theorem}
\newtheorem{lemma}{Lemma}
% definition-like stuff
\theoremstyle{definition}
\newtheorem*{defn}{Definition}
\newtheorem{ex}{Beispiel}
% remark-like stuff
\theoremstyle{remark}
\newtheorem*{notation}{Notation}
\newtheorem*{nb}{Bemerkung}

% commands
\newcommand{\powerset}{\mathcal{P}}
\newcommand{\implies}{\Rightarrow}
\newcommand{\sym}{\text{Sym}}
\newcommand{\gl}{\text{GL}}
\newcommand{\abb}{\text{Abb}}
\newcommand{\inv}[1]{\left(#1\right)^{-1}}
\newcommand{\Inv}[1]{#1^{-1}}

\begin{document}

\author{Sebastian Bechtel}
\title{Einführung in die Algebra}
\date{15. April 2015}

\maketitle % erzeugt den Kopf

\section{Gruppen}

\begin{defn}
    Eine (innere) \underline{Verknüpfung} auf einer Menge $M\neq \emptyset$ ist eine Abbildung $M\times M\to M, (a,b)\mapsto a\cdot b$.
\end{defn}

\begin{defn}
    Eine \underline{Gruppe} ist eine Menge $G\neq \emptyset$ zusammen mit einer Verknüpfung $\cdot$, sodass Assoziativität (A), Existenz eines neutralen Elements (N) und Existenz inverser Elemente (I) erfüllt sind. $G$ ist \underline{abelsch}, falls Kommutativität (K) gilt.
\end{defn}

\begin{bsp}
    \begin{enumerate}
        \item $\mathbb{Z}, \mathbb{Q}, \mathbb{R}, \mathbb{C}$ sind abelsche Gruppen mit $+$ als Verknüpfung.
        \item $\mathbb{Q}^*=\mathbb{Q}\setminus \{0\}, \mathbb{R}^*, \mathbb{C}^*$ mit Multiplikation sind abelsche Gruppen.
        \item Für eine Menge $M$ ist $\sym(M)$ ist eine Gruppe, aber nicht abelsch.
    \end{enumerate}
\end{bsp}

\begin{lemma}
    \begin{enumerate}[label=\alph*)]
        \item Das neutrale Element ist eindeutig.
        \item Inverse Elemente sind eindeutig.
    \end{enumerate}
\end{lemma}

\begin{proof}
    \begin{enumerate}[label=\alph*)]
        \item Seien $e,f$ neutrale Elemente, dann gilt $e=ef=f$.
        \item Sei $a\in G$ und $b,b'\in G$ inverse Elemente. Dann gilt $b'=b'e=b'(ab)=(b'a)b=eb=b$. 
    \end{enumerate}
\end{proof}

\begin{notation}
    multiplikativ: $a\cdot b$ oder $ab$, neutrales Element $e$ oder $1$, inverses Element von $a\in G$ ist $\Inv a$.
\end{notation}

\begin{lemma}
    Es sei $\mathcal{G}=(G,\cdot)$ eine Menge mit assoziativer Verknüpfung, einem linksneutralen Element und linksinversen Elementen, dann ist $\mathcal{G}$ eine Gruppe.
\end{lemma}

\begin{proof}
    Sei $a\in G$ und $b\in G$ mit $ba=e$. Nach (I') gibt es $c\in G$ mit $cb=e$. Also $ab=eab=cbab=ceb=cb=e$.

    Sei nun $a\in G$, es gilt $ae=a(\Inv aa)=ea=e$.
\end{proof}

\begin{lemma}
    \begin{enumerate}
        \item $\inv{\Inv{a}}=a$, $\inv{ab}=\Inv b\Inv a$
        \item $ab=ac \implies b=c$ für alle $a,b,c\in G$.
        \item für $a,b\in G$ gibt es genau ein $x\in G$, sodass $ax=b$.
    \end{enumerate}
\end{lemma}

\begin{proof}
    \begin{enumerate}
        \item $\inv{\Inv a}=a$ klar. Für $a,b,c\in G$: $(\Inv b\Inv a)ab=\Inv b(\Inv aa)b=\Inv beb=\Inv bb=e$ (andere Richtung analog)
        \item $ab=ac\implies \Inv a(ab)=\Inv a(ac)\implies b=c$
        \item Setze $x=\Inv ab$, dann erhält man $ax=a(\Inv ab)=(a\Inv a)b=eb=b$
    \end{enumerate}
\end{proof}

\begin{defn}
    Sei $a\in G$, $(G,\cdot)$ Gruppe. Für $n\in \mathbb{Z}$ definiere:
    
    $$a^0:=e, \quad a^n:=a^{n-1}a \quad \text{ für } n\geq 1$$
    $$a^n:=\left(\Inv a\right)^{-n} \quad \text{ für } n < 0$$
\end{defn}

\begin{lemma}
    Für $a\in G$ gilt: $a^n a^m=a^{n+m}=a^m a^n$, $\left(a^m \right)^n = a^{n\cdot m}$, $ab=ba \implies \left(ab \right)^n = a^n b^n$
\end{lemma}

\begin{bsp}
    \begin{enumerate}
        \item $K$ Körper, dann ist $\gl_n(K)$ ein Gruppe bzgl. Matrixmultiplikation.
        \item $M\neq \emptyset$ Menge, $(G, \cdot)$ Gruppe, definiere $\abb(M,G):=G^M$. Für $f,g\in \abb(M,G)$ ist $f\cdot g$ gegeben durch $(f\cdot g)(m)=f(m)\cdot g(m)$ für $m\in M$. Dann ist $(\abb(M,G), \cdot)$ eine Gruppe.
    \end{enumerate}
\end{bsp}

\section{Untergruppen}

\begin{defn}
    Sei $(G, \cdot)$ Gruppe. Eine Teilmenge $H\subset G$ heißt Untergruppe von $G$, falls $(H, \cdot)$ eine Gruppe ist.

    Äquivalent dazu:

    \begin{enumerate}[label=(\roman*)]
        \item Für $a,b\in H$ gilt $ab\in H$ (Abgeschlossenheit)
        \item $e\in H$
        \item für $a\in H$ ist $\Inv a \in H$
    \end{enumerate}
\end{defn}

\begin{thm}
    Sei $(G, \cdot)$ Gruppe und $H\subset G$ nicht-leer. Dann gilt: $H$ induziert Untergruppe von $(G, \cdot)$ gdw. $a\Inv b\in H$ für $a,b\in H$.
\end{thm}

\begin{proof}
    "$\Rightarrow$" \surd

    "$\Leftarrow$" 

    \begin{itemize} 
        \item $a=b \implies e\in H$
        \item $e,a\in H \implies e\Inv a\in H \implies \Inv a \in H$
        \item $a,\Inv b\in H \implies a\inv{\Inv b} \in H \implies ab\in H$
    \end{itemize}
\end{proof}

\end{proof}

\end{document}
