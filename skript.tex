\documentclass[12pt]{scrartcl}%{article} % Beginn der LaTeX-Datei
\usepackage{mathtools,amsmath,amssymb,amsthm}  % erleichtert Mathe 
\addtolength{\jot}{0.2em}
\usepackage{enumitem}% schicke Nummerierung
\newcommand{\sbt}{\,\begin{picture}(-1,1)(-1,-3)\circle*{3}\end{picture}\ }
\renewcommand{\labelitemi}{\sbt}

\usepackage{graphicx} % für Grafik-Einbindung
\usepackage{float} 
%\usepackage[dvips]{hyperref}
\usepackage{tikz-cd}
\tikzset{ % das braucht man für die kommutativen Diagramme, wenn man babel german benutzt!!
  every picture/.append style={
    execute at begin picture={\shorthandoff{"}},
    execute at end picture={\shorthandon{"}}
  }
}
\usepackage{nicefrac}

% environments
\newtheorem{thm}{Satz}
\newtheorem{lemma}{Lemma}
\newtheorem{kor}{Korollar}
% definition-like stuff
\theoremstyle{definition}
\newtheorem*{defn}{Definition}
\newtheorem{ex}{Beispiel}
% remark-like stuff
\theoremstyle{remark}
\newtheorem*{notation}{Notation}
\newtheorem*{nb}{Bemerkung}

% commands
\newcommand{\powerset}{\mathcal{P}}
\newcommand{\sym}{\text{Sym}}
\newcommand{\gl}{\text{GL}}
\newcommand{\abb}{\text{Abb}}
\newcommand{\inv}[1]{\left(#1\right)^{-1}}
\newcommand{\Inv}[1]{#1^{-1}}

\usepackage[ngerman]{babel}
\usepackage[autostyle=true,german=quotes]{csquotes}
%\usepackage[T1]{fontenc}
%\usepackage{lmodern}
% Einstellungen, wenn man deutsch schreiben will, z.B. Trennregeln
%\usepackage[applemac]{inputenc}
\usepackage[utf8]{inputenc}  % für Unix-Systeme

% TODO Bspw. ord wird zZ als drei vAriablen (o, r, d) gesetzt. Zu einem vernünftigen Operator ändern?
% TODO Anführungszeichen sind bei mir alle falsch. Funktioniert das bei euch?
% TODO Einige Beweise sind schrecklich strukturiert und zum Teil rettbar fehlerhaft. Darf ich?

\begin{document}

\author{Sebastian Bechtel, Isburg Knof, Theresa Tran}
\title{Einführung in die Algebra}
\date{15. April 2015}

\maketitle

\section{Gruppen}

\begin{defn}
    Eine (innere) \underline{Verknüpfung} auf einer Menge $M\neq \emptyset$ ist eine Abbildung $M\times M\to M, (a,b)\mapsto a\cdot b$.
\end{defn}

\begin{defn}
    Eine \underline{Gruppe} ist eine Menge $G\neq \emptyset$ zusammen mit einer Verknüpfung $\cdot$, sodass Assoziativität (A), Existenz eines neutralen Elements (N) und Existenz inverser Elemente (I) erfüllt sind. $G$ ist \underline{abelsch}, falls Kommutativität (K) gilt.
\end{defn}

\begin{ex}
    \begin{enumerate}
        \item $\mathbb{Z}, \mathbb{Q}, \mathbb{R}, \mathbb{C}$ sind abelsche Gruppen mit $+$ als Verknüpfung.
        \item $\mathbb{Q}^*=\mathbb{Q}\setminus \{0\}, \mathbb{R}^*, \mathbb{C}^*$ mit Multiplikation sind abelsche Gruppen.
        \item Für eine Menge $M$ ist $\sym(M)$ ist eine Gruppe, aber nicht abelsch.
    \end{enumerate}
\end{ex}

\begin{lemma}
    \begin{enumerate}[label=\alph*)]
        \item Das neutrale Element ist eindeutig.
        \item Inverse Elemente sind eindeutig.
    \end{enumerate}
\end{lemma}

\begin{proof}
    \begin{enumerate}[label=\alph*)]
        \item Seien $e,f$ neutrale Elemente, dann gilt $e=ef=f$.
        \item Sei $a\in G$ und $b,b'\in G$ inverse Elemente. Dann gilt $b'=b'e=b'(ab)=(b'a)b=eb=b$. 
    \end{enumerate}
\end{proof}

\begin{notation}
    multiplikativ: $a\cdot b$ oder $ab$, neutrales Element $e$ oder $1$, inverses Element von $a\in G$ ist $\Inv a$.
\end{notation}

\begin{lemma}
    Es sei $\mathcal{G}=(G,\cdot)$ eine Menge mit assoziativer Verknüpfung, einem linksneutralen Element und linksinversen Elementen, dann ist $\mathcal{G}$ eine Gruppe.
\end{lemma}

\begin{proof}
    Sei $a\in G$ und $b\in G$ mit $ba=e$. Nach (I') gibt es $c\in G$ mit $cb=e$. Also $ab=eab=cbab=ceb=cb=e$.

    Sei nun $a\in G$, es gilt $ae=a(\Inv aa)=ea=a$.
\end{proof}

\begin{lemma}
    \begin{enumerate}
        \item $\inv{\Inv{a}}=a$, $\inv{ab}=\Inv b\Inv a$
        \item $ab=ac$ impliziert $b=c$ für alle $a,b,c\in G$.
        \item für $a,b\in G$ gibt es genau ein $x\in G$, sodass $ax=b$.
    \end{enumerate}
\end{lemma}

\begin{proof}
    \begin{enumerate}
        \item $\inv{\Inv a}=a$ klar. Für $a,b\in G$: $(\Inv b\Inv a)ab=\Inv b(\Inv aa)b=\Inv beb=\Inv bb=e$ (andere Richtung analog) % FIXME Welche andere Richtung?
        \item $ab=ac$ impliziert $\Inv a(ab)=\Inv a(ac)$ impliziert $b=c$
        \item Setze $x=\Inv ab$, dann erhält man $ax=a(\Inv ab)=(a\Inv a)b=eb=b$.
       	Die Eindeutigkeit folgt aus Punkt 2. \qedhere
    \end{enumerate}
\end{proof}

\begin{defn}
    Sei $a\in G$, $(G,\cdot)$ Gruppe. Für $n\in \mathbb{Z}$ definiere:
    
    $$a^0:=e, \quad a^n:=a^{n-1}a \quad \text{ für } n\geq 1$$
    $$a^n:=\left(\Inv a\right)^{-n} \quad \text{ für } n < 0$$
\end{defn}

\begin{lemma}
    Für $a\in G$ gelten $a^n a^m=a^{n+m}=a^m a^n$ und $\left(a^m \right)^n = a^{n\cdot m}$.
    $ab=ba$ impliziert $\left(ab \right)^n = a^n b^n$.
\end{lemma}

\begin{ex}
    \begin{enumerate}
        \item $K$ Körper, dann ist $\gl_n(K)$ ein Gruppe bzgl. Matrixmultiplikation.
        \item $M\neq \emptyset$ Menge, $(G, \cdot)$ Gruppe, definiere $\abb(M,G):=G^M$. Für $f,g\in \abb(M,G)$ ist $f\cdot g$ gegeben durch $(f\cdot g)(m)=f(m)\cdot g(m)$ für $m\in M$. Dann ist $(\abb(M,G), \cdot)$ eine Gruppe.
    \end{enumerate}
\end{ex}



\section{Untergruppen}

\begin{defn}
    Sei $(G, \cdot)$ Gruppe. Eine Teilmenge $H\subset G$ heißt Untergruppe von $G$, falls $(H, \cdot)$ eine Gruppe ist.

    Äquivalent dazu:

    \begin{enumerate}[label=(\roman*)]
        \item Für $a,b\in H$ gilt $ab\in H$ (Abgeschlossenheit)
        \item $e\in H$
        \item Für $a\in H$ ist $\Inv a \in H$
    \end{enumerate}
\end{defn}

\begin{thm}
    Sei $(G, \cdot)$ Gruppe und $H\subset G$ nicht-leer. Dann gilt: $H$ induziert Untergruppe von $(G, \cdot)$ gdw. $a\Inv b\in H$ für $a,b\in H$.
\end{thm}

\begin{proof}
    "$\Rightarrow$" \checkmark

    "$\Leftarrow$" 
    \begin{itemize} 
        \item $a=b$ impliziert $e\in H$
        \item $e,a\in H$ impliziert $e\Inv a\in H$ impliziert $\Inv a \in H$
        \item $a,\Inv b\in H$ impliziert $a\inv{\Inv b} \in H$ impliziert $ab\in H$ \qedhere
    \end{itemize}
\end{proof}

\begin{ex}
    \begin{enumerate}[label=(\alph*)]
        \item $\{e\}, G$ induzieren je eine Untergruppe für alle Gruppen $(G,\cdot)$.
        \item $K$ Körper. $\text{SL}_n(K)=\{A\in M_n(K): \det(A)=1\}$ induziert Untergruppe von $\gl_n(K)$, die spezielle lineare Gruppe.
    \end{enumerate}
\end{ex}

\begin{defn}
    Eine Untergruppe heißt \underline{echt}, falls sie nicht trivial ist. % FIXME Welche sind trivial? {e} und G?
\end{defn}

\begin{lemma}
	Es sei $(H_{j})_{j \in J}$ eine Familie von Untergruppen $H_{j} \subset G$. Dann ist $\bigcap_{j \in J}H_{j}$ eine Untergruppe von G.
\end{lemma}

\begin{proof}
	Übung
\end{proof}

\begin{defn}
	Es sei $M$ eine Teilmenge von $G$. Die \underline{von $M$ erzeugte Untergruppe} ist der Durchschnitt aller Untergruppen, die $M$ enthalten.
\end{defn}

\begin{notation}
	$\langle M \rangle =\bigcap_{M \subset H \subset G}H$, wobei $H$ Untergruppe
\end{notation}

\begin{nb}
	\begin{enumerate}[label=(\alph*)]
		\item $\langle \emptyset \rangle = \lbrace e \rbrace$
		\item Für $M \neq \emptyset$ gilt: $\langle M \rangle = \lbrace m_{1}^{\varepsilon_{1}} \cdot ... \cdot m_{n}^{\varepsilon_{n}} : m_{1},...,m_{n} \in M, \varepsilon_{1},...,\varepsilon_{n} \in \lbrace -1,+1\rbrace, n \geq 0 \rbrace$
		\item Für $M = \lbrace g \rbrace$ gilt: $\langle g \rangle = \lbrace g^{n} : n \in \mathbb{Z} \rbrace$. Von g erzeugte zyklische Untergruppe von G.
	\end{enumerate}
\end{nb}

\begin{defn}
	$G$ heißt \underline{zyklisch}, falls $G = \langle g \rangle$ für ein $g \in G$ gilt. \newline
	Ist $G = \langle M \rangle$ mit $M$ endlich, so heißt $G$ \underline{endlich erzeugt}.
\end{defn}

\begin{defn}
	\begin{enumerate}[label=(\roman*)]
		\item Die \underline{Ordnung einer Gruppe} $G$ ist $ord(G)=\vert G \vert$.
		\item Die \underline{Ordnung eines Elements} $g \in G$ ist $ord(g)=ord(\langle g \rangle)$.
		\item Ist $ord(g)$ endlich, dann hat g \underline{endliche Ordnung}.
	\end{enumerate}
\end{defn}

\begin{notation}
    $(n,s)$ bezeichnet den größten gemeinsamen Teiler.
\end{notation}

\begin{thm}
	Sei $G$ Gruppe, $g \in G$
	\begin{enumerate}
		\item g hat nicht endliche Ordnung $\Longleftrightarrow$ alle Potenzen von g sind verschieden
		\item g hat endliche Ordnung $\Longleftrightarrow$ $\exists m>0: g^{m}=e$ \newline Dann gilt:
			\begin{enumerate}[label=(\alph*)]
				\item $n := ord(g) = \min \lbrace m>0 : g^{m}=e \rbrace$
				\item $g^{m}=e \Longleftrightarrow m=nk$ für ein $k \in \mathbb{Z}$
				\item $\langle g \rangle = \lbrace e, g^{1},...,g^{n-1}\rbrace$
			\end{enumerate}
		\item $ord(g^{s}) = \frac{n}{(n,s)}$ für $n = ord(g)$ endlich
	\end{enumerate}
\end{thm}

\begin{proof}
	\begin{enumerate}
		\item Wir nehmen an: Für $i,j \in \mathbb{Z}$, oBdA $j>i$ gilt $g^{i}=g^{j}$.
		Dann gilt $g^{j-i}=g^{j}(g^{i})^{-1}=e$.
		Es sei dann n die kleinste positive Zahl, die $g^{n}=e$ erfüllt.
		Sei $m \in \mathbb{Z}$ beliebig.
		Der Divisionsalgorithmus liefert: $m=kn+r$ für $0 \leq r < n$ und $k,r \in \mathbb{Z}$.

		Dann gilt:
		$g^{m}=g^{kn+r}=g^{kn}g^{r}=(g^{n})^{k}g^{r}=eg^{r}=g^{r}$.
		Daraus folgt $\langle g \rangle = \lbrace g^{m} : m \in \mathbb{Z}\rbrace = \lbrace g^{r} : r=0,...,n-1\rbrace$.
		Besonders gilt $ord(g)=n$ ist endlich.
		$\lceil$ Dies zeigt $\Rightarrow$, $\Leftarrow$ klar, dann ist $\langle g \rangle = \lbrace g^{m} : m \in \mathbb{Z}\rbrace$ unendlich $\rfloor$
		\item Alle $g^{r}$ mit $0 \leq r \leq n-1$ sind verschieden, da:
		$g^{i}=g^{j} \Rightarrow g^{j-i}=e \Rightarrow j-i = kn$ mit $k\in\mathbb{Z} \Rightarrow i \equiv j \pmod{n} \Rightarrow i=j$ falls $0\leq i,j\leq n-1$.
		Dies liefert $g^{r}$ mit $0 \leq r \leq n-1$ sind paarweise verschieden und es gilt: $ord(g)=n$. $\lceil a$ und $c\rfloor$
		Aus dem Divisionsalgorithmus folgt (b): $g^{m}=e \Leftrightarrow e=g^{kn+r}=g^{r}$ mit $m = kn+r, 0 \leq r < n \Leftrightarrow r=0$.
		Also $m=kn$ mit $k \in \mathbb{Z}$. % FIXME Scoping-Katastrophe.
		\item Es sei $m=ord(g^{s}), n =ord(g)$.
		Aus $(g^{s})^{m}=e$ folgt (siehe 2), dass $n$ ein Teiler von $sm$ ist.
		Dies liefert: $\frac{n}{(s,n)} \vert \frac{s}{(s,n)}m$. Somit $\frac{n}{(s,n)} \vert m$.
		Nun möchten wir noch zeigen: $m \vert \frac{n}{(s,n)}$. $(g^{s})^{\frac{n}{(s,n)}} = (g^{n})^{\frac{s}{(s,n)}}=e^{\frac{s}{(s,n)}}=e$.
		Daraus folgt $m \vert \frac{n}{(s,n)}$ (wegen 2).
		Also gilt $m = \frac{n}{(s,n)}$. \qedhere
	\end{enumerate}
\end{proof}

\begin{lemma}
	Wir können alle Untergruppen einer zyklischen Gruppe beschreiben mit $G = \langle g \rangle, H \subset G$, es sei $h \in H, h \neq e$. Dann gilt: $h = g^{k}$. % FIXME Aussage des Lemmas unverständlich. Kann den Beweis nicht überprüfen.
\end{lemma}

\begin{proof}
	Wir setzen: $m = min \lbrace k>0 : g^{k} \in H \rbrace$. \newline $\lceil$ Existiert: $G= \langle g \rangle = \langle g^{-1} \rangle, h = g^{k}, k<0$, dann ersetzen wir $h$ durch $h^{-1} \rfloor$ \newline Wir wollen zeigen: $\langle g^{m} \rangle = H$ 
	\begin{enumerate}
		\item $\langle g^{m} \rangle \subset H$ gilt wegen $g^{m} \in H$
		\item Es sei $j \in  \mathbb{Z}$ mit $g^{j} \in H$. Divisionsalgorithmus liefert $j=lm+r$ mit $0 \leq r < m$: $g^{j} \in H \Rightarrow g^{r}=g^{-lm}g^{lm+r}=(g^{m})^{-l}g^{j}$. Also $g^{r} \in H$. Aus der Minimalität von M folgt $r=0$. Dies liefert $g^{j}=(g^{m})^{l} \in \langle g^{m} \rangle$ und somit gilt: $H \subset \langle g^{m} \rangle$ und die zwei Untergruppen stimmen überein.
	\end{enumerate}
\end{proof}

Ähnlich kann man zeigen:

\begin{thm}
	Alle Untergruppen einer zyklischen Gruppe sind zyklisch. Ist $ord(G)=n$ endlich und $m$ ein Teiler von $n$, so ist $H = \langle g^{\frac{n}{m}}\rangle$ die einzige Untergruppe der Ordnung $m$. % FIXME Beweis?
\end{thm}

\begin{defn}
	Sei $H$ eine Untergruppe der Gruppe $G$. Dann kann man die folgende Äquivalenzrelation definieren: \newline $(x,y) \in G^{2}: x \sim_{H} y \Leftrightarrow x = yh$ für ein $h \in H$ \newline $\lceil$Äquivalenzrelation wegen Gruppenaxiomen für $H\rfloor$
\end{defn}

\begin{defn}
	Die Äquivalenzklassen bzgl. $\sim_{H}$ heißen \underline{Linksnebenklassen}.
\end{defn}

\begin{notation}
	Für $a \in G, aH = \{ah : h \in H\}$
\end{notation}

\begin{nb}
	Es gelten folgende Eigenschaften: 
	\begin{itemize}
		\item Die Abbildung $H \rightarrow aH, h \mapsto ah$ ist eine Bijektion. Besonders gilt: $\vert aH \vert = \vert H \vert$ für alle $a \in G$. \newline $\lceil$Die Abbildung ist bijektiv, da sie umkehrbar ist: $aH \rightarrow H, b \mapsto a^{-1}b$ ist die Umkehrfunktion$\rfloor$
		\item $aH \neq bH \Rightarrow aH \cap bH = \emptyset$, d.h. sie sind disjunkt. \newline $\lceil x \in aH \neq bH \Rightarrow x = ah_{1} = bh_{2}$ für $h_{1},h_{2} \in H \Rightarrow a=bh_{2}h_{1}^{-1} \in bH \Rightarrow ah= b(h_{2}h_{1}^{-1}h) \in bH$ für alle $h \in H \Rightarrow aH \subset bH$. Ähnlich gilt $bH \subset aH$. Daraus folgt $aH=bH.\rfloor$ % TODO Äquivalenzklassen sind immer disjunkt. Wieso der Aufwand?
	\end{itemize}
\end{nb}

\begin{defn}
	$G/H = \lbrace aH : a \in G \rbrace$ ist die \underline{Menge der Linksnebenklassen}. \newline Der Index von $H$ ist die Mächtigkeit von $G/H$, d.h. \underline{Index} [$G:H$]$:=\vert G/H \vert$
\end{defn}

\begin{nb}
	\begin{itemize}
	 	\item $\vert G \vert = [G:H]\vert H\vert$
	 	\item Analog ist $a \sim_{H} b$ mit $a,b \in G \Leftrightarrow a=hb$ für ein $h \in H$ ("rechtsäquivalent bzgl. H") eine Äquivalenzrelation.
		\underline{Rechtsnebenklassen}: $Ha=\lbrace ha : h \in H\rbrace$ mit $a \in G$ \newline Bijektion: Für $a \in G$ $aH \rightarrow Ha, x \mapsto a^{-1}xa$
	\end{itemize}
\end{nb}

\begin{defn}
	$H \backslash G$ ist die \underline{Menge der Rechtsnebenklassen}. Dann gilt: $\vert H \backslash G \vert = \vert G/H \vert$ \newline $\lceil$Bijektion: $H \backslash G \rightarrow G/H, Hb \mapsto b^{-1}H \rfloor$
\end{defn}

\begin{lemma} 
Die Funktion
$$  	H\backslash G   \xrightarrow{f} G/H $$
$$	Hb \rightarrow b^{-1}H$$
ist eine Bijektion

\end{lemma}

\begin{proof}
1) f ist wohldefiniert:\\
Gilt $Hb_{1}=Hb_{2}$ für $b_{1},b_{2} \in G$
das heißt $b_{1} {_{\ H}}\sim b_{2}$ \\
existiert ein $h \in H$ mit $b_1 = hb_2$ \\
$b^{-1}*H = (hb_2)^{-1}*H = b_{2}^{-1}*h*H =  b_{2}^{-1}*H$ 

2) f ist Bijektion, da sie wohldefiniert ist:\\
$$  	G/H   \xrightarrow{g} H\backslash G $$
$$	Ha \rightarrow Ha^{-1}$$
ist wohldefiniert, ähnlich zu (1).\\
$g \circ f = Id_{H \backslash G}$ weil $G \circ f(Hb) = gb^{-1}H=H(b^{-1})^{-1}=Hb$ \\
$f \circ g = Id_{G \slash H}$ analog.
\end{proof}

\begin{thm}
Es seien $H$,$K$ Untergruppen von $G$ mit $K \subset H \subset G$. Dann gilt: $$[G:K]=[H:K]*[G*H]$$
\end{thm}

\begin{proof}
Wir nehmen an, dass $[G:H]=m$ und $[H:K]=n$ endlich sind. 
Dann gibt es $a_1,.....,a_n \in H$, so dass
$H/K=\{ a_1K,...,a_nK\}$ und $b_1,...,b_m \in G$ mit
$G\backslash H=\{ b_1H,...,b_mH\}$

Dann gilt:
$$G=\bigcup\limits_{\begin{subarray}{1}i=1...m \\ j=1...n\end{subarray}} b_ia_jK$$
mit $b_ia_jK$ paarweise disunkt.

$b_{i_{1}}a_{j_{1}}K=b_{i_{2}}a_{j_{2}}K$ mit $i_1,i_2 \in \{1,...,n \}$ und $j_1,j_2 \in \{1,...,m \}$

$\Rightarrow b_{i_{1}}H$ und $b_{i_{2}}H$ sind nicht disjunkt.

$\Rightarrow b_{i_{1}}H=b_{i_{2}}H$

Dann gilt: $b_{i_{1}}a_{j_{1}}K=b_{i_{1}}a_{j_{2}}K$

$\overset{*b^{-1}}{\Rightarrow} a_{j_{1}}K=a_{j_{2}}K \Rightarrow a_{j_{1}}=a_{j_{2}}$ \\
Also: Es gibt genau $m*n$ Linksnebenklassen von $K$ in $G$.

$\bigcup\limits_{\begin{subarray}{1}i=1...m \\ j=1...n\end{subarray}} b_ia_jK =  \bigcup\limits_i b_i( \bigcup\limits_j a_jK)= \bigcup\limits_i b_iH=G$

\end{proof}

\begin{nb}
Der Beweis lässt sich zum Fall von unendlichem Index erweitern. Dies liefert eine Bijektion $G/H \times H/K \to G/K$
\end{nb}

\begin{kor}{Satz von Lagrange}\\
Es gilt $\vert G\vert = [G:H]*\vert H\vert$ für jede Untergruppe  $H$ von $G$. Besonders gilt: $\vert H\vert$ teilt $\vert G\vert$ und $ord(g)$ ist ein Teiler von  $\vert G\vert$ für jedes $g\in G$.
\end{kor}

\begin{proof}
$K= \{ e\}$ im vorigen Satz.
\end{proof}

\begin{ex}
$G$ endlich mit $\vert G\vert=p$ Primzahl. Dann existiert ein $g \in G$ mit $g \neq e$

$ord(g)=p \Rightarrow <g>$ besteht aus genau p Elementen.\\
Also $<g>=G$ und $G$ ist die von $g$ erzeugte zyklische Gruppe.
\end{ex}

\section{Normalen Untergruppen und Gruppenhomomorphismen}

\begin{thm}
Es sei $G$ eine Gruppe unf $H\in G$ eine Untergruppe.

Die folgenden Bedingungen sind äquivalent:\\
(i) es gilt: $bH=Hb$ für alle $b\in G$\\
(ii) es gilt: $b^{-1}Hb=H$ für alle $b\in G$\\
(iii) es gilt: $b^{-1}hb=H$ für alle $b\in G$ und $h\in H$
\end{thm}

\begin{defn}
Eine Untergruppe $H$, die eine der Bedingungen (i)-(iii) erfüllt, nennt man eine normale Untergruppe (oder Normalteiler) von $G$.
\end{defn}

\begin{proof}
(i)$\Rightarrow$(ii)\\
Es sei $b \in G$ mit$bH=Hb$
Dann gilt für alle $x \in b^{-1}Hb$:\\
$x=b^{-1}hb \Rightarrow bx=hb \in Hb=bH$ mit $h \in H$\\
$\Rightarrow$ es gibt ein $h' \in H$ mit $bx=bh'$\\
$\Rightarrow b^{-1}xb=b$ also $x=b^{-1}bh'=h'\in H$\\
$\Rightarrow b^{-1}Hb \subset H$\\
Für $h' \in H$ gilt: $bh' \in bH = Hb$ Daraus folgt: $bh'=hb$ für ein $h \in h$ und
$h'=b^{-1}hh'=b^{-1}hb \in b^{-1}Hb$ also $H \subset b^{-1}Hb$\\
(ii) $Rightarrow$ (iii) ist klar.\\
(iii) $Rightarrow$ (i)\\
Für $b \in G$, $h\in $ gilt:\\
$bh=bh(b^{-1}b)=((b^{-1})^{-1}hb^{-1})b \in Hb$\\
(iii) für $b^{-1} \in G$ $hb=bb^{-1}hb \in bH$\\
Daraus folgt: $bH=Hb$
\end{proof}

\begin{nb}
Ist $N$ normal, so gilt:\\
$(aN)(bN)=abN$
\end{nb}

\begin{proof}
$(aN)(bN)=(Na)(bN)=N(ab)N=(abN)N=abN$
\end{proof}

\textbf{Beispiele:}\\
(1) Die trivialen Untergruppen $\{e\}$, $G$ sind normal.\\
\begin{defn}
Eine Gruppe $G$ sodass $\{e\}$ und $G$ die einzigen normalen Untergruppen sind, nennt man einfache Gruppe.
\end{defn}
(2) jede Untergruppe einer abelschen Gruppe ist normal.\\
(3) $SL_n(K)$ ist normale Untergruppe von $GL_n(K)$ ($K$ Körper)

\begin{defn}
Es seien $(G,\cdot_G)$ und $(H,\cdot_H)$ Gruppen. Ein Gruppenhomomorphismus von $G$ in $H$ ist eine Abbildung 
$f:G\rightarrow H$, so dass gilt:\\
$f(a\cdot_{G}b)=f(a)\cdot_{H}f(b)$ für alle $a,b\in G$ 
\end{defn}

\textbf{Eigenschaften}\\
(i) $f(e_G)=e_H.$\\
Für $g\in G$ beliebig gilt:\\
$f(g)=f(e_{G}\cdot_{G}g)=f(e_G)\cdot_{H}f(g)$\\
$\Rightarrow f(e_G)=f(e_G)\cdot_{H}(f(g)\cdot_{H}f(g)^{-1})=f(g)\cdot_{H}f(g)^{-1}=e_H$\\
(ii) $f(g^{-1})=g(g)^{-1}$ für alle $g\in G$\\
$e_H=f(e_G)=f(g\cdot_G g^{-1})=f(g)\cdot_{H}f(g^{-1}) \Rightarrow f(g^{-1})=f(g)^{-1}$\\
(iii) Seien $a_1: G_1\rightarrow G_2$ und $a_2: G_2\rightarrow G_3$ Gruppenhomomorphismen, 
dann ist $a_1\circ a_2: G_1\rightarrow G_3$ ein Gruppenhomomorphismus.\\

\textbf{Beispiele}\\
(1) Für jedes $g \in G$ ist\\
$\mathbb{Z} \rightarrow G$\\
$n \rightarrow g^n$\\
ein Gruppenhomomorphismus.\\
(2) (Exponentialabbildung) \\
$(\mathbb{R},+)\rightarrow(\mathbb{R}^*,\cdot)$\\
$x\rightarrow e^x$\\
ist ein gruppenhomomorphismus.\\
(3) Für $(G,+)$ ablegt ist\\
$G\rightarrow G$\\
$a\rightarrow na = a+...+a$(n-mal)\\
ein Gruppenhomomorphismus.

\begin{defn}
Ist $N$ eine normale Untergruppe von $G$, so ist $G/N$ eine Gruppe, die man als Faktorgruppe von $G$ bezüglich $N$ bezeichnet (oder von $N$ in $G$).
Die natürliche Projektion\\
$G\rightarrow G/N$\\
$a\rightarrow aN$\\
ist dann ein surjektiver Gruppenhomomorphismus.
\end{defn}

\begin{defn}
$G$, $H$ Gruppen, $f: G\rightarrow H$ ein Gruppenhomomorphismus.\\
Bild von $f$: $im(f)=\{f(g),g\in G\}$\\
Kern von $f$: $ker(f)=\{g\in G,f(g)=e_H\}$
\end{defn}

\begin{lemma}
Für einen Gruppenhomomorphismus $f: G \rightarrow H$ gilt:\\
(i) im(f) ist eine Untergruppe von $H$.//
(ii) ker(f) ist eine normale Untergruppe von $G$.
\end{lemma}

\begin{proof}
(i) folgt aus Eigenschaften (i) und (ii) für Gruppenhomomorphismen.\\
(ii) für $a,b \in ker(f): f(ab^{-1})=f(a)f(b^{-1}=e_H$\\
$\Rightarrow ab^{-1} \in ker(f)$
\end{proof}

\begin{defn}
	A, B, C Gruppen, $ f: G \rightarrow H$ Gruppenhomomorphismus. \\
	Das \underline{Bild von $f$} ist: $im(f) = \{f(g) | g \in G\} \subseteq H$. \\
	Der \underline{Kern von $f$} ist: $ker(f) = \{g \in G | f(g) = e_H\} \subseteq G$.
\end{defn}

\begin{lemma}
	Für jeden Gruppenhomomorphismus $ f: G \rightarrow H$ gilt:
	\begin{enumerate}[label=(\roman*)]
		\item $im(f)$ ist eine Untergruppe von H.
		\item $ker(f)$ ist eine normale Untergruppe von G.
		\item $f$ injektiv (bzw. surjektiv) $ \Leftrightarrow ker(f) = {e_G}$ (bzw. $im(f) = H$).
	\end{enumerate}
\end{lemma}

\begin{proof}
	\begin{enumerate}[label=(\roman*)]
		\item $im(f)$ ist wegen der Definition von Gruppenhomomorphismen unter $\cdot_H$ abgeschlossen, enthält $e_H = f(e_G)$ (Eigenschaft (i)) und die Inversen aller seiner Elemente (Eigenschaft (ii)).
		\item Für alle $a, b \in ker(f)$ gilt:
			$$ f(ab^{-1}) = f(a)f(b^{-1}) = f(a)(f(b))^{-1} = ee^{-1} = e.$$
			Somit ist $ker(f)$ eine Untergruppe. \\
			Für jedes $ g \in ker(f)$ und $x \in G$ gilt:
			$$ f(x^{-1}gx) = f(x^{-1})f(g)f(x) = f(x^{-1}) e f(x) = f(x^{-1})f(x) = f(x^{-1}x) = f(e) = e.$$
			Also ist $x^{-1}gx \in ker(f) $ und somit ist $ker(f)$ ein Normalteiler.
		\item Subjektivität: offentsichtlich. Injektivität: vgl. Übung 2, Aufgabe 1.
	\end{enumerate}	
\end{proof}

\begin{nb}
	\begin{itemize}
		\item Ein bijektiver Homomorphismus wird \underline{Isomorphismus} genannt. Die Umkehrfunktion ist dann wieder ein Isomorphismus.
		\item Ein Gruppenhomomorphismus $G \rightarrow G$ heißt \underline{Endomorphismus von G}.
		\item Ein Gruppenisomorphismus $G \rightarrow G$ heißt \underline{Automorphismus von G}.
	\end{itemize}
\end{nb}

\begin{defn}
	$G/N$ heißt \underline{Faktorgruppe von $N$ in $G$}, wenn $N$ normale Untergruppe von $G$ ist. \\
	Gruppenstruktur: $aN \cdot bN = abN$, $eN = N$ neutrales Element \\
	Die \underline{natürliche Projektion} $\pi : G \rightarrow G/N, a \mapsto aN$ ist ein surjektiver Gruppenhomomorphismus mit $ker(\pi) = N, im(\pi) = G/N$.
\end{defn}

\begin{thm}[Homomorphiesatz]
	Es sei $f: G \rightarrow G'$ ein Gruppenhomomorphismus und N ein Normalteiler von G. Ist N im $ker(f)$ enthalten, so gibt es genau einen Gruppenhomomorphismus $\bar{f}: G/N \rightarrow G'$, so dass das folgende Diagramm kommutiert:

    \[ \begin{tikzcd}
            G \arrow[rr,"f=\bar f\circ \pi"] \arrow[rd,"\pi"] & & G'  \\
                                             & G/N \arrow[ru,"\bar f"] &
    \end{tikzcd} \]
\end{thm}

\begin{nb}
	Es gilt $ker(\bar{f}) = \pi(ker(f))$ und $im(\bar{f}) = im(f)$. 
\end{nb}

\begin{kor}
	Wenn $f: G \rightarrow G'$ ein surjektiver Homomorphismus ist, dann ist $\bar{f}: G/ker(f) \rightarrow G'$ ein Isomorphismus.
\end{kor}

\begin{proof}[Beweis des Satzes]
	Aus dem kommutativen Diagramm folgt, dass $\bar{f}(aN) = f(a)$ sein sollte für jedes $aN \in G/N$. \\
	Wir zeigen zuerst, dass $\bar{f}$ wohldefiniert ist. Es seien $a, b \in G$ mit $aN = bN$. Dann gilt $a \sim_{N} b$ und also $b^{-1}a \in N$. Daraus folgt
	$$f(a) = f(ea) = f(b(b^{-1}a)) = f(b) f(b^{-1}a) \stackrel{b^{-1}a \in N \subseteq ker(f)}{=} f(b)e = f(b).$$
	Die Abbildung $\bar{f}: G/N \rightarrow G'$ existiert somit und ist offensichtlich ein Gruppenhomomorphismus, denn
	$$\bar{f}(abN) = f(ab) = f(a)f(b) = \bar{f}(aN)\bar{f}(bN)$$ for all $a, b \in G$.
\end{proof}

\begin{proof}[Beweis der Bemerkung]
	\begin{itemize}
		\item $aN \in ker(\bar{f}) \stackrel{Def. ~ von ~ \bar{f}}{\Leftrightarrow} a \in ker(f)\stackrel{Def. ~ von ~ \pi}{\Leftrightarrow} aN \in \pi(ker(f))$
		\item $im(f) = im(\bar{f} \circ \pi) = \bar{f}(im(\pi)) \stackrel{\pi ~ surj.}{=} \bar{f}(G/N) = im(\bar{f})$
	\end{itemize}
\end{proof}

\begin{thm}[1. Isomorphiesatz]
	Es seien $G$ eine Gruppe, $H \subseteq G$ eine Untergruppe und $N$ ein Normalteiler von $G$. Dann ist $NH = \{nh | n \in N, h \in H\}$ Untergruppe von $G$ und $N \cap H$ ein Normalteiler von $H$. Ferner ist $H/(N \cap H) \rightarrow (NH)/N, a(N \cap H) \mapsto aN$ ein Isomorphismus.
\end{thm}

\begin{proof}
	Es seien $n_1, n_2 \in N, h_1, h_2 \in H$. Dann gilt:
	$$ (n_1h_1)(n_2h_2)^{-1} = n_1h_1h_2^{-1}n_2^{-1} = n_1(h_1h_2^{-1}n_2^{-1}) = \dotsb $$
	[Da $N$ normal, gilt $h_1h_2^{-1}N = Nh_1h_2^{-1}$. Es gibt also ein $n_3 \in N$ mit $h_1h_2^{-1}n_2^{-1} = n_3h_1h_2^{-1}$.]
	$$ \dotsb = (n_1n_3)(h_1h_2^{-1}) \in NH.$$
	Daraus folgt, dass $NH$ Untergruppe von $G$ ist. \\
	Nun betrachten wir $f: H \rightarrow (NH)/N, a \mapsto aN = Na$. $f$ ist ein surjektiver Gruppenhomomorphismus mit $ker(f) = N \cap H$. Aus dem Homomorphiesatz (+ Korollar) folgt dann, dass $\bar{f}: H/(N \cap H) \rightarrow (NH)/N$ ein Isomorphismus ist.
\end{proof}

\begin{thm}[2. Isomorphiesatz]
	Es seien $M, N$ normale Untergruppen einer Gruppe $G$. Gilt $N \subseteq M$, so ist $M/N$ eine normale Untergruppe von $G/N$ und die Abbildung $(G/N)/(M/N) \rightarrow G/M, (aN)M/N \mapsto aM$ ist ein Isomorphismus.
\end{thm}

\begin{proof}
	$f: G/N \rightarrow G/M, aN \mapsto aM$ (wohldefiniert wegen $N \subseteq M$) ist surjektiver Gruppenhomomorphismus mit $ker(f) = M/N$. Die Aussage folgt dann aus dem Korollar zum Homomorphiesatz.
\end{proof}

\begin{notation}
	Seien $A, B, C$ Gruppen, $\alpha: A \rightarrow B$, $\beta: B \rightarrow C$ Gruppenhomomorphismen.
	\begin{itemize}
		\item Falls $im(\alpha) = ker(\beta)$, so sagt man, dass die Folge $A \stackrel{\alpha}{\longrightarrow} B \stackrel{\beta}{\longrightarrow} C$ bei $B$ \underline{exakt} ist.
		\item $\alpha$ ist surjektiv, falls $A \stackrel{\alpha}{\longrightarrow} B \longrightarrow  \{e\}$ bei $B$ exakt ist. \\
		\begin{tabular}{p{4.3cm}p{.1cm}p{.3cm}l}
			& $b$ & $\mapsto$ & $e$
		\end{tabular}
		\item $\alpha$ ist injektiv, falls $\{e\} \longrightarrow A \stackrel{\alpha}{\longrightarrow} B$ bei $A$ exakt ist. \\
		\begin{tabular}{p{3.1cm}p{.1cm}p{.3cm}l}
			& $e$ & $\mapsto$ & $e_A$
		\end{tabular}
		\item Notation: $\{e\} =: 1$. Eine exakte Folge der Form
		$$ 1 \longrightarrow A \stackrel{\alpha}{\longrightarrow} B \stackrel{\beta}{\longrightarrow} C \longrightarrow 1$$
		(d.h. $\alpha$ injekiv, $\beta$ surjekiv und $im(\alpha) = ker(\beta)$), heißt \underline{kurze exakte Folge}.
	\end{itemize}
\end{notation}

\begin{ex}[für kurze exakte Folgen]
	\begin{itemize}
		\item $1 \longrightarrow N \stackrel{\alpha}{\longrightarrow} G \stackrel{\beta}{\longrightarrow} G/N \longrightarrow 1$ ist exakt für jeden Normalteiler $N$ von $G$.
		\item Für G abelsch ($\{e\} =: 0$): \\
		$0 \longrightarrow H \stackrel{\alpha}{\longrightarrow} G \stackrel{\beta}{\longrightarrow} G/H \longrightarrow 0$ ist exakt für jede Untergruppe $H$ von $G$.
	\end{itemize}
\end{ex}



\section{Produkte von Gruppen}

\begin{defn}
	Es sei $(G_i)_{i \in I}$ eine Familie von Gruppen. Das \underline{äußere direkte Produkt} der Familie ist das kartesische Produkt $\prod_{i \in I}G_i$ mit der Verknüpfung $(a_i)_{i \in I} (b_i)_{i \in I} = (a_ib_i)_{i \in I}$. Neutrales Element: $(e_i)_{i \in I}$ mit $e_i \in G_i$ neutral.
\end{defn}

\begin{notation}
	$G_1 \times G_2 \times \dotsb \times G_n$ für endliche Produkte. \\
	$G_1 \oplus G_2 \oplus \dotsb \oplus G_n$ für endliche Produkte, $G_i$ abelsch [additiv].
\end{notation}

\begin{lemma}
	Für jedes $i_0 \in I$ ist die Teilmenge
	$$ \overline{G_{i_0}} = \{(b_i)_{i \in I} \in \prod_{i \in I}G_i | b_i = e_i \text{ für } i \neq i_0\} $$
	ein Normalteiler von $\prod_{i \in I}G_i$, isomorph zu $G_{i_0}$.
\end{lemma}

\begin{proof}
	$\overline{G_{i_0}}$ ist der Kern des Gruppenhomomorphismus
	$$p_{i_0}: \prod_{i \in I}G_i \rightarrow \prod_{i \in I-\{i_0\}}G_i, (b_i)_{i \in I} \mapsto (b_i)_{i \in I-\{i_0\}}.$$
	Ferner ist
	$$j_{i_0}: G_{i_0} \rightarrow \prod_{i \in I}G_i, a \mapsto (b_i)_{i \in I} \text{ mit } b_{i_0}=a \text{ und } b_i = e_i \text{ für } i \neq i_0.$$
	Das Bild ist $\overline{G_{i_0}}$ und $j_{i_0}$ ist injektiv, weil
	$$(b_i)_{i \in I} = (e_i)_{i \in I} \Leftrightarrow b_i = e_i \text{ für alle } i \in I.$$
\end{proof}

\begin{nb}
	\begin{itemize}
		\item $1 \longrightarrow G_{i_0} \stackrel{j_{i_0}}{\longrightarrow} \prod_{i \in I}G_i \stackrel{p_{i_0}}{\longrightarrow} \prod_{i \in I-\{i_0\}}G_i \longrightarrow 1$ ist kurze exakte Folge.
		\item Der Beweis liefert $\overline{G_{i_0}} \cap \langle \bigcup_{i \in I-\{i_0\}}G_i \rangle = \{e\}$.
	\end{itemize}
\end{nb}

\begin{defn}
    Sei $G$ eine Gruppe und $(N_i)_{i\in I}$ eine Familie von normalen Untergruppen, so dass gilt:

    \begin{enumerate}[label=(\roman*)]
        \item $G=<\cup_{i\in I} N_i>$
        \item $N_{i_0} \cap <\cup_{i\in I \setminus \{i_0\}} N_i> = \{e\}$ für jedes $i_0\in I$.
    \end{enumerate}

    Dann ist $G$ das \underline{innere Produkt} von $(N_i)_{i\in I}$.
\end{defn}

\begin{lemma}
    Es sei $G$ das innere Produkt von $(N_i)_{i\in I}$. Dann gilt $ab=ba$ für $a\in N_i, b\in N_j$ mit $i\neq j$.
\end{lemma}

\begin{proof}
    Man rechnet: $$ab\inv{ba}=ab\Inv{a}\Inv{b}=(ab\Inv{a})\Inv{b}=a(b\Inv{a}\Inv{b}) \in N_i\cap N_j$$ Somit folgt $ab\inv{ba}=e$, also $ab=ba$.
\end{proof}

Wir werden uns demnächst nur mit endlichen Produkten beschäftigen.

\begin{lemma}
    Sei $G$ Gruppe, $N_1,\dots,N_r$ normale Untergruppen von $G$. Dann ist $G$ genau dann das innere Produkt von $N_1,\dots,N_r$, wenn gilt:

    \begin{enumerate}[label=(\roman*)']
        \item $G=N_1\dots N_r$
        \item Die Darstellung $a=n_1\dots n_r$ mit $n_j\in N_j$ ist für jedes $a\in G$ eindeutig bestimmt.
    \end{enumerate}
\end{lemma}

\begin{proof}
    (i) äquivalent (i)': folgt aus $$<\bigcup_{i=1}^r N_i> = \{a_1^{\xi_1}\dots a_k^{\xi_k}: a_j\in N_1\cup\dots \cup N_r, \xi_j = \pm 1\}$$ und der Tatsache, dass die $N_j$ normale Untergruppen sind die für paarweise verschiedene $j\neq j'$ kommutieren.

    (ii)' impliziert (ii): oBdA gelte $i_0=1$. Wir möchten zeigen, dass $$N_1\cap <N_2\cup\dots \cup N_r> = N_1 \cap (N_2\dots N_r)= \{e\}$$ Sei $x\in N_1\cap <N_2\dots N_r>$. Dann gilt $x=n_1\in N_1$ und $x=n_2\dots n_r$ mit $n_i\in N_i$ für $i\geq 2$. Also $x=n_1e\dots e=en_2\dots n_r$, somit $n_i=e$ für alle $i$, also $x=e$.

    (ii) impliziert (ii)': Aus $n_1\dots n_r=n_1'\dots n_r'$ mit $n_i,n_i'\in N_i$ folgt $$\inv{n_1'}n_1=(n_2'\dots n_r')\inv{n_2\dots n_r} = \{e\}$$ Die letzte Gleichheit gilt nach (ii). Also folgt $n_i=n_i'$. Vertauschen der Reihenfolge liefert dies für $i\neq 1$.
\end{proof}

\begin{thm}
    Das innere Produkt $G$ von normalen Untergruppen $N_1,\dots,N_r$ ist zum äußeren Produkt $N_1\times\dots \times N_r$ kanonisch isomorph. Folgerung: Wir brauchen nicht zwischen den $\overline{G_i}=N_i$ und den $G_i$ zu unterscheiden.
\end{thm}

\begin{proof}
    Wir definieren $$N_1\times\dots\times N_r \overset{\pi}{\longrightarrow} G,\quad (n_1,\dots,n_r)\mapsto n_1\dots n_r$$ Da das Bild von $\pi$ die Untergruppe $N_1\dots N_r \overset{(i)'}{=} G$ ist, ist $\pi$ surjektiv. Aus (ii)' folgt, dass $\pi$ auch injektiv ist. $\pi$ ist auch Gruppenhomomorphismus: 
    
    \begin{align*}
        \pi((m_1,\dots,m_r)\cdot (n_1,\dots,n_r)) \\
        =\pi((m_1n_1,\dots,m_rn_r))=m_1n_1\dots m_rn_r=m_1\dots m_rn_1\dots n_r \\
        \pi(\bar m)\pi(\bar n)
    \end{align*}
    
    Dabei gilt die vorletzte Gleichheit, weil wir Elemente aus verschiedenen Untergruppen vertauschen dürfen.
\end{proof}

\begin{defn}
    Für $N,H$ Gruppen ist eine Gruppenerweiteruung von $N$ durch eine kurze, exakte Folge von Gruppen
    
    \[ \begin{tikzcd}
        1 \arrow[r] & N \arrow[r] & G \arrow[r] & H \arrow[r] & 1
    \end{tikzcd} \]
    
    gegeben, wobei $i$ injektiv, $p$ surjektiv, $\text{im } i = \ker p$ ($\nicefrac{G}{H} \cong H$).

\end{defn}

\begin{ex}
    $G=N\times H$ Diese ist i.A. nicht die einzige Gruppenerweiterung.
\end{ex}

\begin{defn}
    Es seien $G$ eine Gruppe, $N\subseteq G$ Normalteiler von $G$ und $H\subseteq G$ beliebige Untergruppe, so dass gilt: $$G=NH \quad \text{und} \quad N\cap H = \{e\}$$ Dann ist $G$ das \underline{semidirekte Produkt} von $N$ und $H$, Notation: $G=N\rtimes H$.
\end{defn}

\begin{nb}
    Jedes $a\in G$ hat eine eindeutige Darstellung $a=nh$ mit $n\in N, h\in H$. Dies liefert eine Bijektion $$N\times H \to G=N\rtimes H, (n,h)\mapsto nh$$
\end{nb}

\begin{thm}
    Für jedes $h\in H$ ist die Konjugationsabbildung $$\gamma_h: N\to N, \quad n\mapsto hn\Inv{h}$$ ein Automorphismus von $N$. Dies ergibt den Homomorphismus $$H\overset{\gamma}{\longrightarrow} \text{Aut}(N), h\mapsto \gamma_h$$
\end{thm}

\begin{proof}
    $\gamma$ ist Homomorphismus:

    \begin{gather*}
        \gamma_{h_1h_2}(n)=(h_1h_2)n\inv{h_1h_2}=h_1h_2n\Inv{h_2}\Inv{h_1} \\
        = h_1\gamma_{h_2}(n)\Inv{h_1}=(\gamma_{h_1}\circ\gamma_{h_2})(n)
    \end{gather*}

    für $n\in N$.
\end{proof}

\begin{nb}
    \begin{enumerate}[label=(\arabic*)]
        \item Das semidirekte Produkt von $N$ und $H$ wird von $\gamma: H\to \text{Aut}(N)$ eindeutig festgelegt. Es gilt nämlich: $$(n_1h_1)(n_2h_2)=n_1(h_1n_2\Inv{h_1})h_1h_2=n_1\gamma_{h_1}(n_2)h_1h_2$$ wobei $$n_1\gamma_{h_1}(n_2)\in N, h_1h_2\in H$$.
        \item Umgekehrt definiert $\gamma: H\to \text{Aut}(N)$ immer ein semidirektes Produkt $N\rtimes H$:

            \begin{gather*}
                G=(N\times H, \cdot_\gamma) \\
                (n_1,h_1)\cdot_\gamma (n_2,h_2)=(n_1\gamma_{h_1}(n_2), h_1h_2)
            \end{gather*}

            ist Gruppe mit neutralem Element $(e_N, e_H)$ und inversen Elementen $$\Inv{(n,h)}=(\gamma_{\Inv{h}}(\Inv{n}),\Inv{h})$$

            Definiere:

            \begin{gather*}
                N^*\coloneqq = \{(n, e_H): n\in N\} \cong N, \\
                H^*\coloneqq = \{(e_N, h): h\in H\} \cong H
            \end{gather*}

            $N^*,H^*$ sind Untergruppen von $G$, isomorph zu $N$ bzw. $H$. $$\pi: G\longrightarrow H, \quad (n,h)\mapsto h$$ ist ein Homomorphismus mit $\ker \pi = N^*$, also ist $N^*$ normale Untergruppe.

            Das $N^*\cap H^* = \{e\}$ ist klar. $$(n,h)=(n,e_H)\cdot_\gamma (e_N, h)$$ liefert $G=N^*H^*$.
    \end{enumerate}
\end{nb}

\begin{defn}
    Eine kurze exakte Folge 

    \[ \begin{tikzcd}
        1 \arrow[r] & N \arrow[r, "i"] & G \arrow[r, "p"] & H \arrow[l, bend left, dashed, "s"] \arrow[r] & 1
    \end{tikzcd} \]

    spaltet, falls ein Homomorphismus $s: H\to G$ existiert, mit $p\circ s = \text{id}_H$. $p$ heißt ein \underline{Schnitt} von $p$.
\end{defn}

Für $G=N\rtimes H$ Gruppenerweiterung:

\[ \begin{tikzcd}
        1 \arrow[r] & N \arrow[r,"i"] & G \arrow[r,"p"] & H \arrow[r] & 1
\end{tikzcd} \]

wobei $i: N\to G, n\mapsto (n,e_H), p:G\to H, (n,h)\mapsto h$ ist $j:H\to G, h\mapsto (e_N,h)$ Schnitt. $H$ ist Untergruppe. Man rechnet $(p\circ j)(h)=p(e_N,h)=h$

\begin{thm}
    Es sei

    \[ \begin{tikzcd}
        1 \arrow[r] & N \arrow[r,"i"] & G \arrow[r,"p"] & H \arrow[r] & 1
    \end{tikzcd} \]

    eine Gruppenerweiterung, die mit einem Schnitt $s:H\to G$ spaltet. Dann ist $G$ das semidirekte Produkt von $N$ und $H$, definiert durch $$\gamma: H\to \text{Aut}(N), \quad h\mapsto \gamma_h$$
\end{thm}

\begin{proof}
    Wir setzen $\rho: N\rtimes H \to G, (n,h)\mapsto i(n)s(h)$. Zeige, dass $\rho$ Homomorphismus ist:

    \begin{gather*}
        \rho((n_1,h_1)\cdot (n_2,h_2)) = \rho(n_1\gamma_{h_1}(n_2), h_1h_2) \\
        = i(n_1)i(\gamma_{h_1}(n_2))s(h_1)s(h_2) = i(n_1)s(h_1)i(n_2)\Inv{s(h_1)}s(h_1)s(h_2) \\
        = \rho(n_1,h_1)\rho(n_2,h_2).
    \end{gather*}
    Es bleibt zu zeigen, dass $\rho$ bijektiv ist.
    
    
    
    % VL 19.05.
    \begin{enumerate}
    \item{Injektivität}
    
    Sei $(n,h)\in \ker(\rho)$. Dann gilt

    $i(n)\cdot s(h) = e_G \Rightarrow e_H = p(e_G) = p(i(n)\cdot s(h)) = p(i(n))\cdot p(s(h))$.

    Aber $p(i(n)) = e_H$ wegen der kurzen exakten Folge und $p(s(h)) = h$. Also $e_H = h$.

    Damit ist $i(n)\cdot s(h) = i(n)\cdot s(e_H) = i(n)\cdot e_G = i(n)$. Da $(n,h)\in\ker(\rho)$ folgt $i(n) = e_G$ und somit $n=e_N$, da $i$ injektiv ist.

    Daraus folgt $\ker(\rho) = \{(e_N,e_H)\}$. Somit ist $\rho$ injektiv.
    \item{Surjektivität}
    
    Sei $a\in G$ beliebig. Wir setzen: $b := a\cdot\inv{s(p(a)}$. Wegen der Schnitteigenschaft von $s$ ist $p\circ s = \text{id}_H$ und somit

    $p(b) = p(a)\cdot (p\circ s)(\Inv{p(a)})=p(a)\cdot\Inv{p(a)}=e_G$. Daraus folgt $b\in\ker(p)= \text{im}(i)$. Und somit $\exists n\in N$ mit $i(n) = b$. Dies liefert:

    $\rho(n,p(a)) = b\cdot s(p(a))=a\cdot s(p(a))^{-1}s(p(a))=a$.
    \end{enumerate}
    
\end{proof}

\begin{ex}

\begin{enumerate}[label=(\alph*)]
    \item Für einen beliebigen Körper $K$ zerfällt (bzw. spaltet)
    \[ \begin{tikzcd}
         1 \arrow[r] & SL_n(K) \arrow[r] & GL_n(K) \arrow[r,"\det"] & K^* \arrow[r] & 1.
    \end{tikzcd} \]
    Das heißt $GL_n(K) = SL_n(K)\rtimes K^*$.
    
    \item Es sei $N$ eine beliebige abelsche Gruppe und $H := (\{\pm 1\},\cdot)\left[\cong(\mathbb{Z}/2\mathbb{Z},+)\right]$. Für $h\in H$ sei $\gamma_h$ definiert als $\gamma_h:N\to N, \gamma_h(n) := n^h$.
    
    Dann heißt $D_n := N \rtimes H$ verallgemeinerte Diedergruppe. Diese hat die Verknüpfung 
    $(n_1,\epsilon_1)\cdot(n_2,\epsilon_2) = (n_1 n_2^{\epsilon_1},\epsilon_1\epsilon_2)$. Es ist $N \cong N\times\{1\}$ eine normale Untergruppe von $D_N$ mit Index $[D_N:N] = 2$.
    
    \item Spezialfall: $N = \mathbb{Z}/k\mathbb{Z}$, $D_{2k} := D_{\mathbb{Z}/k\mathbb{Z}}$ heißt Diedergruppe der Ordnung $2k$. Dann ist $D_{2k}$ die Symmetriegruppe des regelmäßigen k-Gons. Die Untergruppe $N\subset D_{2k}$ ist gerade die Menge der Drehungen.
    
    Ist $G$ eine von zwei Elementen $n,h\in G$ erzeugt, also $G = \left< n,h\right>$, mit $ord(n) = k$ ($\cong$ Drehung) und $ord(h) = 2$ ($\cong$ Spiegelung) und außerdem $h^{-1}nh=n^{-1}$, so gilt $G\cong D_{2k}$.
\end{enumerate}

\end{ex}

\section{Die symmetrische Gruppe}

\begin{defn}
    Für eine Menge $M\neq\emptyset$ bildet $S_M$ als Menge aller Bijektionen von $M\to M$ mit Verkettung von Funktionen als Verknüpfung und der Identität als neutrales Element eine Gruppe, die \emph{symmetrische Gruppe von $M$}. Die Elemente von $S_M$ heißen \emph{Permutationen}.
\end{defn}

\begin{thm}[Satz von Cayley]
    Jede Gruppe $G$ ist isomorph zu einer Untergruppe von $S_G$.
\end{thm}
\begin{proof}

    Für jedes Element $a\in G$ ist die Linkstranslation um $a$, also $\lambda_a:G\to G, b\mapsto a\cdot b$ eine Permutation.

    Die Abbildung $\lambda:G\to S_G, a\mapsto \lambda_a$ ist ein Homomorphismus, denn für $a,b,h\in G$ gilt

\[\lambda_{ab}(h) = abh = a(bh) = a\lambda_b(h) = \lambda_a(\lambda_b(h)) = (\lambda_a\circ\lambda_b)(h).\]

    Es sei $a\in \ker(\lambda)$. Dann gilt $\lambda_a = \text{id}_G \Rightarrow ah = h$ für alle $h\in G $ und damit
    \[a = a\cdot e = \lambda_a(e) = e \Rightarrow \ker(\lambda) = \{e\}.\]
    
    Also ist $\lambda$ injektiv und somit $G\cong\text{im}(\lambda)$.
\end{proof}

\begin{defn}
    Es sei $M\neq\emptyset$ eine Menge und $\sigma$ eine Permutation von $M$.
    \begin{enumerate}
        \item Der Träger von $\sigma$ ist $\text{supp}(\sigma) := \{m\in M: \sigma(m)\neq m\}$.
    
        \item Ein Zyklus der Länge $l$ ist eine Permutation $\sigma\in S_M$ mit $l = |supp(\sigma)|$, so dass
    \[supp(\sigma) = \{m_1,m_2,\dots,m_l\}\text{ mit }\sigma(m_j) = m_{j+1},\; 1\leq j<l,\; \sigma(m_l) = m_1.\]
    Für einen solchen Zyklus schreiben wir $\sigma = (m_1m_2\dots m_l)$.
    
        \item 2-Zyklen werden Transpositionen genannt.
    \end{enumerate}
\end{defn}


\begin{nb}
    Sind $\sigma,\tau\in S_M$ mit $supp(\sigma)\cap supp(\tau) = \emptyset$, so gilt $\sigma\circ\tau = \tau\circ\sigma$.
\end{nb}


\begin{defn}
    Im Spezialfall $M=\{1,2,\dots,n\}$ schreibt man $S_n := S_M$. Ein Element $\sigma\in S_n$ schreibt man als \[\sigma = \begin{pmatrix}
     1 & 2 & \dots & n \\
     \sigma(1) & \sigma(2) & \dots & \sigma(n) \end{pmatrix}.\]
\end{defn}

\begin{ex}
     $\sigma = \begin{pmatrix}
     1 & 2 & 3 & 4 \\
     4 & 2 & 1 & 3 \end{pmatrix} \in S_4$.
    $\sigma$ ist der 3-Zyklus $(143) = (431) = (314)$.
\end{ex}


\begin{thm}
\begin{enumerate}[label=(\alph*)]
    \item Jede Permutation $\sigma\in S_n, \sigma\neq Id$, kann als Produkt von Zyklen mit disjunkten Trägern geschrieben werden. Diese Darstellung ist bis auf die Reihenfolge der Faktoren eindeutig bestimmt.
    \item Für einen beliebigen l-Zyklus $(m_1m_2\dots m_l)\in S_n$ und $\sigma\in S_n$ gilt 
    \[\sigma\circ(m_1m_2\dots m_l)\circ\sigma^{-1} = (\sigma(m_1)\sigma(m_2)\dots\sigma(m_l)).\]
    \item Die symmetrische Gruppe $S_n$ wird von den Transpositionen erzeugt.
\end{enumerate}
\end{thm}

\begin{proof}
\begin{enumerate}[label=(\alph*)] % (a), (b), (c)
    \item Es sei $\sigma\in S_n, \sigma\neq\text{id}$. Für $m_1,m_2\in\{1,\dots,n\}=:M$ definieren wir 
    \[m_1\sim m_2 :\Leftrightarrow m_1 = \sigma^l(m_2)\qquad\text{für ein }l\in\mathbb{Z}.\]
    Dann ist $\sim$ eine Äquivalenzrelation auf $M$. Es sei $M = K_1\cup K_2\cup\dots\cup K_r$ die Zerlegung von $M$ in disjunkte Äquivalenzklassen. Wir betrachten $a_j\in K_j$ beliebig. Es sei $l_j$ die kleinste positive Zahl mit $\sigma^{l_j}(a_j) = a_j$. Das liefert uns
    \[a_j,\sigma(a_j),\sigma^2(a_j),\dots,\sigma^{l_j-1}(a_j)\in K_j\]
    und diese Elemente sind paarweise verschieden. Aus der Definition folgt, dass $\sigma\upharpoonright_{K_j}=(a_j\sigma(a_j)\dots\sigma^{l_j-1}(a_j))\in S_{K_j}$ ein Zyklus der Länge $l_j$ ist. Wenn wir dies für alle $j=1,\dots,r$ tun, erhalten wir:
    \[\sigma = (a_1\dots\sigma^{l_1-1}(a_1))(a_2\dots\sigma^{l_2-1}(a_2))\dots(a_r\dots\sigma^{l_1-r}(a_r)).\]
    Dies ist genau die gesuchte Darstellung, wenn man zudem $(a_j):=\text{id}$ für $l_j = 1$ definiert.

    Eindeutigkeit: Es sei für $s\in\mathbb{N}$ $\sigma = \tau_1\tau_2\dots\tau_s$ eine andere Darstellung. Dann ist $\sigma\upharpoonright_{supp(\tau_k)} = \tau_k$ für alle $k$ ein Zyklus. Damit ist $supp(\tau_k) = K_j$ für ein $j=1,\dots,r$ und somit ist $\tau_k$ gerade ein solcher Zyklus wie in unserer Konstruktion. Diese ist also eindeutig.

    \item Es sei $\sigma\in S_n$ und $(m_1\dots m_k)$ ein Zyklus. Dann gilt
    \[(\sigma(m_1\dots m_k)\Inv\sigma)(\sigma(m_{\alpha}))=(\sigma(m_1\dots m_k))(m_{\alpha}) = \sigma(m_{\alpha+1})\text{ mit }m_{k+1}:=m_1.\] 
    Für $m\notin \{m_1,\dots,m_k\}$ gilt $(\sigma(m_1\dots m_k)\sigma^{-1})(\sigma(m))=(\sigma(m_1\dots m_k))(m) = \sigma(m)$. Daraus folgt $\sigma\circ(m_1m_2\dots m_l)\circ\sigma^{-1} = (\sigma(m_1)\sigma(m_2)\dots\sigma(m_l))$.

    \item Wegen (a) genügt es zu zeigen, dass für alle $k\in\mathbb{N}$ alle k-Zyklen Produkte von Transpositionen sind. Für einen beliebigen Zyklus gilt
    \[(m_1\dots m_k) = (m_1m_2)(m_2m_3)\dots (m_{k-1}m_k).\]
    Beispiel: $(1234) = (12)(23)(34)$. Graphisch veranschaulicht:
    \[\begin{pmatrix}
        1 & 2 & 3 & 4 \\ % (34)
        1 & 2 & 4 & 3 \\ % (23)
        1 & 3 & 4 & 2 \\ % (12)
        2 & 3 & 4 & 1
    \end{pmatrix}.\]
    
    Anmerkung: Es reichen deutlich weniger Transpositionen aus, um $S_n$ zu erzeugen.
\end{enumerate}

\end{proof}

\begin{defn}
Für $\sigma\in S_n$ ist das Signum von $\sigma$ definiert als
    \[sgn(\sigma) := \prod_{i<j}\frac{\sigma(i)-\sigma(j)}{i-j}\qquad(\in\{\pm 1\}).\]
Permutationen $\sigma\in S_n$ mit $sgn(\sigma) = 1$ (bzw. $-1$) heißen gerade (bzw. ungerade) Permutationen.
\end{defn}

\begin{lemma}
Die Abbildung $sgn:S_n\to\{\pm 1\}$ ist ein Gruppenhomomorphismus.
\end{lemma}

\begin{proof}
Es seien $\sigma,\tau\in S_n$ beliebig. Dann gilt 

\begin{equation*}
    \begin{split}
        sgn(\sigma\tau) &= \prod_{i<j}\frac{\sigma\tau(i)-\sigma\tau(j)}{i-j} = \prod_{i<j}\frac{\sigma\tau(i)-\sigma\tau(j)}{\tau(i)-\tau(j)}\cdot\prod_{i<j}\frac{\tau(i)-\tau(j)}{i-j} \\
        &= \left(\prod_{\substack{i<j,\\\tau(i)<\tau(j),\\ i':=\tau(i),j':=\tau(j)}}\frac{\sigma\tau(i)-\sigma\tau(j)}{\tau(i)-\tau(j)}\cdot \prod_{\substack{i<j,\\\tau(j)<\tau(i),\\ i':=\tau(j),j':=\tau(i)}}\frac{\sigma\tau(i)-\sigma\tau(j)}{\tau(i)-\tau(j)}\right)\cdot sgn(\tau) \\
        &=\prod_{i'<j'}\frac{\sigma(i')-\sigma(j')}{i'-j'}\cdot sgn(\tau) \\
        &= sgn(\sigma)\cdot sgn(\tau).
    \end{split}
\end{equation*}
\end{proof}

=======
% VL 26.5.
\newpage
	\begin{kor}
	Jede gerade Permutation ist das Produkt einer geraden Anzahl von Transpositionen. Jede ungerade Permutation ist Produkt einer ungeraden Anzahl von Transpositionen.
	\end{kor}

	\begin{ex}
	m-Zyklus $(n_1n_2...n_m)$ ist das Produkt von $m-1$ Transpositionen.  
	
	$\Rightarrow sgn((n_1n_2....n_m))=(-1)^{m-1}$
	\end{ex}

	\begin{kor}
	$\sigma = (m_1....m_{k_1})(m_{k_1+1}...m_{k_2})...(m_{k_{r-1}}....m_{k_r})$	$\Rightarrow sgn(\sigma)=(-1)^{k_r-r}$
	\end{kor}

	\begin{defn}
	Die normale Untergruppe $A_n=ker(sgn)$ der geraden Permutationen wird die alternierende Gruppe genannt.
	\end{defn}
	
	\begin{nb}
	Index von $A_n$ ist $S_n$ ist 2 $\Rightarrow ord(A_n)=\frac{1}{2}ord(S_n)$\\
	$ord(S_n)=n(n-1)...1=n!$
	\end{nb}
	
	\begin{lemma}
	$A_n$ wird durch die 3-Zyklen von$S_n$ erzeugt.
	\end{lemma}
	
	\begin{proof}
	Es sei $\sigma \in A_n$ eine gerade Permutation. Wir schreiben $\sigma$ als Produkt von Transpositionen:
	$\sigma =(a_1b_1)(a_2b_2)....(a_nb_n)$\\
	Für disjunkte Träger gilt:\\
	$(a_{2_{j-1}b_{2_{j-1}}})(a_{2_j}b_{2_j})=(a_{2_{j-1}}a_{2_j})(a_{2_j}b_{2_{j-1}})(a_{2_{j-1}}a_{2_j})(a_{2_j}b_{2_j})=(a_{2_{j-1}}a_{2_j}b_{2_{j-1}})(a_{2_{j-1}}a_{2j}b_{2j})$\\
	Produkt von 3-Zyklen.
	
	Wenn es ein gemeinsames Element gibt: ob $b_{2_{j-1}}=a_{2_j}$\\
	$(a_{2_{j-1}}a_{2_j})(a_{2_j}b_{2_j})=(a_{2_{j-1}}a_{2j}b_{2j})$ e-Zyklus\\
	Für $\{ a_{2_{j-1}}b_{2_{j-1}}\} =\{ a_{2_j}b_{2_j} \} : (a_{2_{j-1}}b_{2_{j-1}})(a_{2_j}b_{2_j})=id$\\
	Daraus folgt: $\sigma$ ist das Produkt von 3-Zykeln.
	\end{proof}
	
	\begin{ex}
	\begin{enumerate}
		\item $S_1=\{ (1)\}$
		\item $S_2=\{ id, (12)\}$ zyklisch der Ordnung 2
		\item $S_3=\{ id,(12),(13);(23),(123),(132)\} \supsetneq A_3=\{ id,(123),(132)\}$
		\item $S_3=A_3\rtimes \mathbb{Z}/2\mathbb{Z} \cong  \mathbb{Z}/3\mathbb{Z} \rtimes \mathbb{Z}/2\mathbb{Z} = D_6$ Diedergruppe mit 6 Elementen.
	\end{enumerate}
	\end{ex}
	 
\section{Gruppenhoperationen}
	\begin{defn}
	Es sei $(G, \bullet)$ eine Gruppe und $X$ eine Menge. Eine Operation von G auf X ist eine Abbildung:
	$$G \times X \rightarrow X$$
	$$(g,x \rightarrow g\bullet x)$$
	welche folgende Bedingungen erfüllt:
	\begin{enumerate}
		\item $e \bullet x = x$ ($e\in G$ neutral)
		\item $(a\bullet b)\bullet x = a\bullet (b\bullet x) \forall a,b\in G, x\in X$
	\end{enumerate}
	\end{defn}
	
	\begin{ex}
	$V$, $K$-Vektorraum, die allgemeine lineare Gruppe $GL_k(V)$ operiert auf $V$:
	$$GL_k(V) \times X \rightarrow V$$
	$$(A,v)\rightarrow Av$$
	\end{ex}

	\begin{notation}
	Wenn $G$ auf $X$ operiert, sagt man, dass $X$ eine $G$-Menge ist. Bezeichnung: $G\circlearrowright X$
	\end{notation}

	\begin{nb}
	Für jedes Element $g\in G$ existiert
	$$\tau_g: X \rightarrow X$$
	$$x \rightarrow gx$$
	\end{nb}
	
	$\tau_g$ ist eine Bijektion mit Inverse $\tau_{g^{-1}}$ Aus 1. und 2. folgt:
	$$\tau G \rightarrow S_X$$
	$$g \rightarrow \tau_g$$
	
	ist ein Gruppen homomorphismus. Umgekehrt definiert jeder Gruppenhomomorphismus $\tau: G \rightarrow S_X$ durch 
	$g: x=(\tau(g))(x)$ eine Gruppenoperation.
	
	\begin{defn}
	\begin{enumerate}
		\item Es sei $\tau: G \times X \rightarrow X$ eine Gruppenoperation. Für $x\in X$ heißt $Gx=\{ gx : g \in G\}$ die Bahn (oder Orbit) von X. 
		\item Die Untergruppe $G_x=\{ g\in G : gx=x\}$ heißt Isotropiegruppe (oder Stabilisator) von $X$.
		\item Menge der Fixpunkte unter $G$: $X^G=\{ x \in X : gx=x \forall g \in G\}$
		\item Die Gruppenoperation ist treu, falls $G \rightarrow S_X$ injektiv ist.
		\item Die Gruppenoperation ist transitiv, wenn $X$ eine Bahn ist.
	\end{enumerate}
	\end{defn}
	
	\begin{ex}
	\begin{enumerate}
		\item Der Satz von Cayley ergibt einen Invektiven Homomorphismus.
		$$G \rightarrow S_G$$
		$$a \rightarrow \lambda_a$$
		wobei
		$$\lambda_a: G \rightarrow G$$
		$$b \rightarrow ab$$
		Das heißt: 
		$$(G\times G) \rightarrow G$$
		$$(a,b) \rightarrow ab$$
		ist eine treue Operation von G auf G.
		Diese Operation ist transitiv: es gilt $eG=G$, d.h. $G$ ist eine Bahn.
		
		\item Die Rechtstranslation
		$$\varphi : G \rightarrow G$$
		$$b \rightarrow ba^{1}$$
		für beliebiges $a\in G$ definiert eine treue und transitive Gruppenoperation von $G$ auf $G$:
		$$(G,G) \rightarrow G$$
		$$a \bullet b := ba^{-1}$$	
		
		\item $G$ operiert auf sich selbst durch Konjugation:
		$$a /in G i_a : G \rightarrow G$$
		$$x \rightarrow axa^{-1}$$	
		Die Konjugation mit $a$ ist immer ein Automorphismus von G.
		$$G \xrightarrow{i} Aut(G) \subset S_G$$
		$$a \rightarrow i_a$$
		ein gruppenhomomorphismus, d.h. eine Operation von $G$ auf sich selbst.
		\begin{proof}
		$a,b \in G$ $i_{ab}(x)=abx(ab)^{-1}=abxb^{-1}a^{-1}$\\
		$=a(i_b(x))a^{-1}=i_a\circ i_b(x) \forall x\in G$
		\end{proof}
		\begin{nb}
		Die Bahnen von $i$ heißen Konjugationsklassen. Der Kern von $i$ heißt Zentrum von $G$.\\
		$Z(G)=ker(i)\{ a \in G : axa^{-1}=x$ $\forall x \in G\} = \{a \in G : ax =xa$ $\forall x \in G\}$\\
		Es gilt $G^G=\{ y \in G : byb^{-1}=y$ $\forall b \in G\}=\{ y \in G : by =yb$ $\forall b \in G\}=Z(G)$\\
		
		D.h. $Z(G)$ ist auch die Menge der Fixpunkte. Der Stabilisator von $x \in G$:\\
		$G_x=\{a \in G : axa^{-1}=x\}=\{ a \in G : ax = xa\}=:C_G(x)$
		\end{nb}		
	\end{enumerate}
	\end{ex}
	
	\begin{thm}
	$G \times X \rightarrow X$ Gruppenoperation. Dann ist die Abbildung
	$$p: G/G_x \rightarrow Gx$$
	$$gG_x \rightarrow gx$$
	ein bijektiver $G$-Homomorphismus für alle $X \in X$.
	\end{thm}
	
	\begin{defn}
	$X$,$Y$ $G$-Mengen $f: X \rightarrow Y$ ist ein $G$-Homomorphismus, falls $f(gx)=g(f(x))$
	\end{defn}
	
	\begin{kor}
	$|Gx|=|G/G_x|=[G:G_x]$ Index
	\end{kor}
	
	\begin{proof}
	$p$ ist wohldefiniert: $\forall a,b,x \in G$ gilt:
	$$aG_x=bG_x \Leftrightarrow b^{-1}a \in G_x \Leftrightarrow b^{-1}ax=x \Leftrightarrow ax=bx \Leftrightarrow p(a)=p(b)$$
	D.h. $p$ ist wohldefiniert als auch bijektiv.
	$p$ ist ein $G$-Homomorphismus: $\forall a,b \in G$:
	$$p(a(b(G_x)))=p(ab(G_x))=abx=ap(bG_x)$$
	Die Bahnen einer $G$-Operation sind die Äquivalenzklassen bezüglich der Äquivalenzrelation:
	
	$x,y \in X : x \sim y \xLeftrightarrow{Def} x=gy$ für ein $g\in G$\\
	D.h. die Bahnen bilden eine Partition von $X$.
	
	Für ein Repräsentantensystem $\{ x_i\}_{i\in I}$ der Bahnen von G gilt die Bahnengleichung:

	$$|X|=\sum\limits_{\begin{subarray}{1}i \in I\end{subarray}}|Gx_i|=\sum\limits_{\begin{subarray}{1}i \in I\end{subarray}}[G:G_{x_i}]=|X^G|+\sum\limits_{\begin{subarray}{1}i \in I\\ x_i \notin X^G\end{subarray}}[G:G_{x_i}]$$
	
	\end{proof}

%VL 2.6.15
Für die Operation von G auf sich selbst durch Konjugation gilt:
\begin{thm} [Klassengleichung]
Ist $\{x_{i}\}_{i \in I}$ ein Repräsentantensystem für die Konjugationsklassen der Gruppe G, so gilt: 
\begin{gather*}
|G|= |Z(G)|+\sum_{x_{i}\notin Z(G)}[G:C_{G}(x_{i})] \\ 
Z(G)=\{a \in G :ag=ga\ \ \forall g \in G\}\ \ Zentrum \\
C_{G}(x_{i})= \{a \in G :ax_{i}=x_{i}a\}\ \ Zentralisator \\
\end{gather*}
Für endliche Gruppen: $[G:G_{x_{i}}]=\dfrac{|G|}{|G_{x_{i}}|}$
\end{thm}

\begin{nb}
Anwendung: Existenz von Fixpunkten
\end{nb}

\begin{defn} 
Eine endliche Gruppe G wird \underline{p-Gruppe} genannt, wobei p eine Primzahl ist, wenn die Ordnung von G eine Potenz $p^n$ von p ist $(n\geq 1)$.
\end{defn}

\begin{thm}
Das Zentrum einer p-Gruppe G enthält nicht triviale Elemente $a \neq e$.
\end{thm}

\begin{proof}
\begin{enumerate}
\item \underline{n=1:} ord(G)=p ist eine Primzahl. Dann ist G eine zyklische Gruppe. Da G abelsch ist, gilt dann: G=Z(G).
\item \underline{$n\geq 2$:} Aus dem Satz von Lagrange folgt, dass die Ordnung jeder Untergruppe H von G ein Teiler von $p^n$ ist, d.h. eine Primpotenz $p^k$ mit $0\leq k\leq n$. Es seien $x_{1},...,x_{m}$ Repräsentanten der Konjugationsklassen von G. \\
$C_{G}(x_{i})$ Untergruppe $\Rightarrow |C_{G}(x_{i})|=p^k_{i}$ mit $0\leq k_{i} \leq n$ und $k_{i}=n \Leftrightarrow x_{i} \in Z(G)$  \\
Klassengleichung: $p^n=|X|=|Z(G)|+\sum_{x_{i}\notin Z(G)}p^{n-k_{i}}$ mit $n-k_{i}>0$ \\
$|Z(G)|= p^n-\sum_{x_{i}\notin Z(G)}p^{n-k_{i}}$ ist durch p teilbar. Wegen $|Z(G)|\geq 1$ gilt $|Z(G)|\geq p$. Es gibt also mindestens p-1 Elemente in Z(G), die vom neutralen Element verschieden sind.
\end{enumerate}
\end{proof}

\begin{thm}
Es sei G eine p-Gruppe, die auf einer endlichen Menge X operiert. Dann gilt:
\begin{equation*}
|X^G|\equiv |X|(mod\ \ p)
\end{equation*}
\end{thm}

\begin{proof}
Es sei $\{x_{1},...,x_{m}\}$ ein Repräsentantensystem der Bahnen. \\
$|C_{G}(x_{i})|=p^k_{i}$ mit $0\leq k_{i} \leq n$ (aus dem Satz von Lagrange); $k_{i}=n\Leftrightarrow x_{i} \in X^G$ \\
Die Bahnengleichung liefert $|X|-|X^G|=\sum_{x_{i}\notin Z(G)}p^{n-k_{i}}$ ist ein Vielfaches von p.
\end{proof}

\section{Strukturtheorie der Gruppen}

\subsection{Die sylowschen Sätze}

Satz von Lagrange $\Rightarrow ord(H)|ord(G)$ für jede Untergruppe H von G
Aber: wenn h ein Teiler von ord(G), existiert dann auch H mit ord(H)=h?

\begin{thm}
Es sei G eine abelsche Gruppe der Ordnung $pn$. Dann existiert immer ein $a \in G$ der Ordnung p. 
\end{thm}

\begin{proof}
Induktion auf $n=\dfrac{|G|}{p}$
\begin{itemize}
\item \underline{Basis:} n=1, d.h. $|G|=p$. Daraus folgt: $G=<a>$ ist zyklisch und der Erzeuger a hat Ordnung p.
\item \underline{Induktionsschritt:} Wir nehmen an, dass die Aussage für eine Gruppe der Ordnung $pm^,$ mit $m^,<n$ gilt.\\
Es sei $n\neq e$ ein beliebiges Element von G, und $m:=ord(h)$. Wenn m durch p teilbar ist, enthält die zyklische Untergruppe $<h>$ ein Element der Ordnung p. \\
Wenn p kein Teiler von m ist, so ist p ein Teiler von $|G/<n>|=\dfrac{|G|}{|<n>|}=\dfrac{pn}{m}<pn$. Nach Induktionsvoraussetzung enthält die Gruppe $|G/<n>|$ ein Element $a<n>$ der Ordnung p. Es sei $k:=ord(a)$ die Ordnung der gewählten Repr. ($a \in G$). \\
$(a<n>)^k=a^k<n>=<n>$ (neutrales Element) liefert $ p = \dfrac{ord(a<n>)}{k}$. \\
Daraus folgt, dass $a^{\dfrac{k}{p}}$ ein Element von $<a>$ der Ordnung p ist.
\end{itemize}
\end{proof}

\begin{thm} [Cauchy]
Ist p eine Primzahl, die die Ordnung der endl. Gruppe G teilt, so enthält G ein Element der Ordnung p.
\end{thm}

\begin{proof}
Induktion auf $ord(G)$
\begin{itemize}
\item \underline{Basis:} ord(G)=1 ist klar
\item \underline{Induktionsschritt:} Wir nehmen an, dass die Aussage für alle Gruppen T gilt mit $ord(T)<ord(G)$. \\
Wenn G eine Untergruppe H enthält, deren Ordnung durch p teilbar ist, so enthält H ein Element der Ordnung p. \\
Wir können annehmen, dass die Ordnung jeder Untergruppe $H\neq G$ nicht durch p teilbar ist. \\
Im Spezialfall der Zentralisation gilt: $p \nmid ord(C_{G}(a))$ für $a \notin Z(G)$ \\
Klassengleichung: $|G|=|Z(G)|+\sum_{x_{i}\notin Z(G)}\dfrac{|G|}{|C_{G}(a)|}$ impliziert $p|\mid |Z(G)|$. Aus der Annahme folgt dann Z(G)=G. \\
G ist also abelsch und die Aussage folgt aus dem vorherigen Satz.
\end{itemize}
\end{proof}

\begin{kor}
Die Ordnung einer endlichen Gruppe G ist genau dann eine Primpotenz für eine Primzahl p, wenn die Ordnung jedes Elements von G eine Primpotenz bezüglich p ist.
\end{kor}

\begin{proof}
\begin{itemize}
\item[$"\Rightarrow"$]ord(g) teilt $ord(G)=p^n$ (Lagrange) impliziert, dass $ord(g)=p^k$ gilt
\item[$"\Leftarrow"$]Ist die Ordnung von G von der Form $ord(G)=p^\alpha q$ mit $q\neq 1$ prim zu p. Dann enthält G ein Element der Ordnung p für jeden Primfaktor von q, somit ist die Ordnung nicht von der Form $p^k$. Widerspruch!
\end{itemize}
\end{proof}

\begin{nb}
(unendliche) Gruppen G, für die gilt: $ord(a)=p^{n(a)}\ \ \forall a \in G$, p feste Primzahl, werden \underline{p-Gruppen} genannt.
\end{nb}

\begin{defn}
Es sei p eine Primzahl, G eine endliche Gruppe. Man schreibt $|G|=p^n q$ mit $n \in \mathbb{N}$ und q teilerfremd zu p. Eine Untergruppe S von G ist eine \underline{p-Sylowuntergruppe}, wenn $|S|=p^n$ gilt.
\end{defn}

\begin{thm} [Sylow]
Es sei G eine endliche Gruppe der Ordnung $|G|=p^n q$ mit p Primzahl, q teilerfremd zu p. Dann gilt:

\begin{enumerate}
\item G enthält eine Untergruppe der Ordnung $p^\alpha$ für jedes $1\leq \alpha \leq n$.
\item Es sei H eine p-Untergruppe von G und S eine p-Sylowuntergruppe. Dann existiert ein $g \in G$ mit $H \subset gSg^{-1}$.
\item Die Anzahl der p-Sylowuntergruppen ist ein Teiler von q von der Form $s=1+kp$ mit $k \in \mathbb{N}$.
\end{enumerate}
\end{thm}

\begin{proof}
\begin{enumerate}
\item Induktion auf $ord(G)$ \\
Wir nehmen an, dass die Aussage für alle Gruppen der Ordnung $< ord(G)$ gilt. \\
Wenn G eine echte Untergruppe H enthält, wobei $|H|$ durch p teilbar ist, existiert nach der Induktionsvor. eine Untergruppe von H von der Ordnung $p^\alpha$.(fertig) \\
Wir können uns auf den Fall beschränken, dass $p^n \nmid ord(H)$ für alle Untergruppen $H\neq G$gilt. \\
$(a_{i}){i \in I}$ Repräsentantensystem der Konjugationsklassen. \\
$p^n q=|G|=|Z(G)|+\sum_{x_{i}\notin Z(G)}\dfrac{|G|}{|C_{G}(a_{i})|}$ \\
$p^n \nmid |C_{G}(a_{i})|$ ergibt sich aus der Annahme $p^n \mid |G|$. \\
Daraus folgt: $\dfrac{|G|}{|C_{G}(a_{i})|}$ ist durch p teilbar. Daraus folgt, dass $|Z(G)|$ durch p teilbar ist. Aus dem Satz von Cauchy folgt dann, dass $b \in Z(G)$ mit $ord(b)=p$ existiert.

% VL 09.06.
Da $b$ im Zentrum von $G$ liegt, ist $\langle b \rangle$ eine normalte Untergruppe von $G$. Wir betrachten die Faktorgruppe $G/ \langle b \rangle$. Es gilt
	$$ |G/ \langle b \rangle | = \frac{|G|}{| \langle b \rangle |} = p^{n-1}q < p^nq. $$
	Nach Induktionsvoraussetzung enthält $G / \langle b \rangle$ eine Untergruppe $\bar{S}$ der Ordnung $p^{\alpha - 1}$. Es sei $S = \{ x \in G : x \langle b \rangle \in \bar{S} \}$. $S$ ist eine Untergruppe von G und es gilt $ \bar{S} \cong S/ \langle b \rangle$, denn für die natürliche Projektion $p: S \rightarrow \bar{S} \text{ gilt } ker(p) = \langle b \rangle$ und somit ist der Homomorphiesatz anwendbar. Das liefert
	$$ p^{\alpha - 1} = |\bar{S}| = \frac{|S|}{| \langle b \rangle |} = \frac{|S|}{p}. $$
	Somit gilt $|S| = p^{\alpha}$ und $S$ ist die gesuchte $p$-Gruppe.
	
	\item Sei $H$ ein $p$-Untergruppe und $S$ ein $p$-Sylowuntergruppe. $H$ operiert auf die Menge $X = G/S$ durch
	$$ H \times X \rightarrow X, h \cdot gS := (hg)S.$$
	Es gilt $|X| = \frac{|G|}{|S|} = \frac{p^nq}{p^n} = q$. Da $H$ eine $p$-Gruppe ist, gilt $|X^H| = |X| = q \ (modp)$. Da $q$ prim zu $p$ ist, ist $|X^H|$ nicht durch $p$ teilbar. Insbesondere kann $X^H$ nicht leer sein, denn $p|0$. Es gibt also ein $g \in G$ mit $gS \in X^H$. Für alle $h \in H$ gilt dann $(hg)S = gS$ und somit $g^{-1}hg \in S$ und $h \in gSg^{-1}$. Dies liefert, dass $H$ in $gSg^{-1}$ enthalten ist.
	
	\item siehe unten.
\end{enumerate}
\end{proof}

\begin{kor}
	Dei $p$-Sylowuntergruppen bilden eine Klasse von konjugierten Untergruppen von $G$.
\end{kor}

\begin{proof}
	Sei $S$ ein $p$-Sylowuntergruppe d.h. $|S| = p^n$. Dann ist $gSg^{-1}$ ein Untergruppe der Ordnung $p^n$ (für $g \in G$ beliebig), d.h. $gSg^{-1}$ ist eine $p$-Sylowuntergruppe. Es sei $S'$ ein beliebige Sylowuntergruppe von $G$. Aus 2. folgt: $S' \subseteq gSg^{-1}$ für ein $g \in G$. $|S'| = p^n = |gSg^{-1}|$ impliziert $S' = gSg^{-1}$.
\end{proof}

Bevor wir den dritten Teil des Sylow-Satzes beweisen, wenden wir uns folgender allgemeiner Frage zu: Wie bestimmt man die Anzahl der zu einer gegebenen Untergruppe konjugierten Untergruppen? Es sei $\mathcal{U}$ die Menge aller Untergruppen von G und $ G \times \mathcal{U} \rightarrow \mathcal{U}, g \cdot H \mapsto gHg^{-1} $ eine Gruppenoperation.

\begin{defn}
	Der Stabilisator einer Untergruppe $H$ von $G$ unter dieser Operation heißt \underline{Normalisator} von $H$: $N_G(H) = \{ g \in G: gHg^{-1} = H \}$. \\
	Die zu $H$ konjugierten Untergruppen bilden die Bahn $G \cdot H$ von $H$. Es gilt also
	$$ |G \cdot H| = | \{ A \in \mathcal{U} : A = gHg^{-1} \text{ mit } g \in G \} | = [G:N_G(H)] .$$
	Für eine endliche Gruppe $G$ ist diese Anzahl immer ein Teiler von $|G|$.
\end{defn}

Für den Beweis des dritten Teils benötigen wir folgendes Lemma:

\begin{lemma}\label{lem:Sylow}
	Sei $T$ eine $p$-Sylowuntergruppe von $G$ und $H$ eine $p$-Untergruppe. Wenn $H \subseteq N_G(T)$ gilt, gilt bereits $H \subseteq T$.
\end{lemma}

\begin{proof}
	Da $T$ eine normale Untergruppe von $N_G(T)$ ist und $H$ eine Untergruppe von $N_G(T)$ ist, kann man den ersten Isomorphiesatz anwenden: $TH = \{ gh : g \in T, h \in H \}$ ist eine Untergruppe von $N_G(T)$ und es gilt $H/(T \cap H) \cong TH/T$. Daraus folgt
	$$ [TH : T] = [H : T \cap H] = \frac{|H|}{|T \cap H|} = \frac{p^{\alpha}}{|T \cap H|} = p^{\alpha '} \text{ mit } 0 \leq \alpha ' \leq \alpha.$$
	$TH/T$ ist also eine $p$-Gruppe. Es gilt: $|TH/T| = 1 \Leftrightarrow T \cap H = H \Leftrightarrow H \subseteq T$. \\
	Die Inklusionskette $T \subseteq TH \subseteq G$ von Untergruppen ergibt
	$$[G:T] = [G:TH][TH:T] \Leftrightarrow q = [G:TH]p^{\alpha '} \text{, d.h. } p^{\alpha '} | q.$$
	Da $q$ nicht durch $p$ teilbar ist, muss $p^{\alpha '} = 1$ sein. Also ist $H$ in $T$ enthalten.
\end{proof}

\begin{proof}[Beweis des dritten Teils des Sylow-Satzes]
	Sei $X$ die Menge aller $p$-Sylowuntergruppen von $G$, $S \in X$ beliebig (Existenz: siehe Teil 1), $s := |X|$. Es gilt $s = [G:N_G(S)]$. \\
	Aus der Inklusionskette $S \subseteq N_G(S) \subseteq G$ folgt
	$$[N_G(S):S] [G:N_G(S)] = [G:S] \Leftrightarrow [N_G(S):S] s = q \text{, d.h. } s | q.$$
	Wir lassen $S$ wie folgt auf X operieren:
	$$S \times X \rightarrow X, g \cdot T \mapsto gTg^{-1}.$$
	$S \in X$ ist ein Fixpunkt dieser Operation. Wir möchten zeigen, dass dieser der einzige Fixpunkt ist. Da $S$ eine $p$-Gruppe ist, gilt dann: $s = |X| = |X^S| = 1 \pmod{p}$ (was zu zeigen war).
	\begin{align*}
		T \in X^S &\Leftrightarrow &gTg^{-1} = T \text{ für alle } g \in S \\
		&\Leftrightarrow &g \in N_G(T) \text{ für alle } g \in S \\
		&\Leftrightarrow &S \subseteq N_G(T)
	\end{align*}
	Aus $S \subseteq N_G(T)$ folgt mit Lemma~\ref{lem:Sylow}, dass $S \subseteq T$ und somit $S = T$ (da $|S| = |T| = p^n$). $S \in X$ ist damit der einzige Fixpunkt.
\end{proof}

\underline{Anwendungen/Folgerungen des Sylow-Satzes}

\begin{thm}
	Es seien $p, q$ Primzahlen mit $p < q$ und $p \nmid q-1$. Dann ist jede Gruppe $G$ der Ordnung $pq$ eine zyklische Gruppe.
\end{thm}

\begin{proof}
	Es sei $s$ die Anzahl der $p$-Sylowuntergruppen von $G$. Aus Teil 3 des Sylow-Satzes folgt $s | q, s \equiv 1 \pmod{p}$. Da $q \not \equiv 1 \pmod{p}$, muss $s = 1$ gelten, denn die Primzahl $q$ ist nur durch $1$ und $q$ teilbar. \\
	Es gibt also eine einzige $p$-Sylowuntergruppe $S$ von $G$. Aus Teil 2 des Sylow-Satzes folgt $gSg^{-1} = S$ für alle $g \in G$. Somit ist $S$ eine normale Untergruppe. \\
	Für die Anzahl $r$ der $q$-Sylowuntergruppen gilt: $r | p$ und $r \equiv 1 \pmod{q}$. Wenn $r = p$, dann gilt $r = p \not = 1 \pmod{q}$ (wegen $p < q$). Dies ist also nicht möglich. Es gilt dann $r = 1$. Es existiert genau eine $q$-Sylowuntergruppe $R$ von $G$ und $R$ ist deswegen ein Normalteiler von $G$. Da $S$ und $R$ normal sind, ist $SR = \{ ab : a \in S, b \in R\}$ eine normale Untergruppe. \\
	$S \cap R = \{ e\}$ (die Elemente müssen Ordnung $1$ haben). Dies ergibt, dass $SR$ das (innere) direkte Produkt von $S$ und $R$ ist. Insbesondere gilt $|SR| = |S \times R| = |S||R| = pq = |G|$. Folglich gilt $G = S \times R$. Man zeigt dann, $G = \langle ab \rangle \text{ mit } \langle a \rangle = S, \langle b \rangle = R$ zyklisch.
\end{proof}

% VL 16.06.
\begin{thm}
    Es existiert keine einfache Gruppe der Ordnung 30.
\end{thm}

\begin{proof}
    Es ist $|G| = 30 = 2\cdot 3\cdot 5$. Es seien $s$ die Anzahl der 5-Sylowuntergruppen und $t$ die der 3-Sylowuntergruppen. Angenommen, $G$ sei einfach. Dann gilt: $s>1$ und $t>1$, denn mit $s=1$ (bzw. $t=1$) wäre die 5-Sylowuntergruppe (bzw. 3-Sylowuntergruppe) normal wegen Konjugiertheit der Sylowgruppen zueinander (Teil 2 im Satz). Aus Teil (3) folgt
        \[ s|6,\; s\equiv 1(5)\Rightarrow s=6.\]
    Also gibt es 6 verschiedene 5-Sylowuntergruppen, jede davon hat Ordnung $5$. Damit enthalten die 5-Sylowuntergruppen zusammen $6\cdot 4 =24$ Elemente und das neutrale Element. Analog gilt für $t$
        \[ t|10,\; t\equiv 1(3) \Rightarrow t=10.\]
    Damit enthalten die 3-Sylowuntergruppen $10\cdot 2=20$ Elemente und das neutrale Element.
    
    Aber: $20+24+1=45>30=\text{ord}(G)$. Also kann $G$ nicht einfach sein.
\end{proof}

\subsection{Normal- und Kompositionsreihen}

\begin{defn}
    Es sei $G$ eine Gruppe. Eine Normalreihe in $G$ ist eine Folge von Untergruppen
        \[ \{e\} = G_0 \subseteq G_1 \subseteq G_2 \subseteq\dots\subseteq G_n=G\qquad n\geq 1\qquad(n=0\text{ für } G=\{e\})\]
    so dass für alle $i,1\leq i\leq n$ $G_{i-1}$ Normalteiler von $G_i$ ist.
\end{defn}

\begin{nb}
    Die $G_i,\;i\leq n-2,$ sind nicht notwendigerweise Normalteiler von $G$.
\end{nb}

\begin{defn}
    Es seien 
        \[\begin{split}
            &\underline{G}: G_0\subseteq G_1 \subseteq\dots\subseteq G_n \\
            &\underline{H}: H_0\subseteq H_1\subseteq\dots\subseteq H_m 
        \end{split}\]
    zwei Normalreihen.
    \begin{itemize}
        \item $\underline{H}$ ist eine Verfeinerung von $\underline{G}$, wenn $\underline{H}$ aus $\underline{G}$ durch das Einfügen von Untergruppen erhalten werden kann.
        \item $\underline{G}$ und $\underline{H}$ heißen äquivalent, wenn $m=n$ gilt und es eine Permutation $\sigma\in S_n$ gibt, so dass 
        \[ G_j/G_{j-1}\cong H_{\sigma(j)}/H_{\sigma(j)-1}\]
            für alle $1\leq j\leq n$ gilt.
    \end{itemize}
\end{defn}

\begin{thm}
Es sei $G$ eine Gruppe, $\underline{G}$ und $\underline{H}$ Normalreihen in $G$. Dann besitzen $\underline{G}$ und $\underline{H}$ äquivalente Verfeinerungen.
\end{thm}

\begin{proof}
    Es seien 
    \[\begin{split}
            &\underline{G}: G_0\subseteq G_1 \subseteq\dots\subseteq G_n \\
            &\underline{H}: H_0\subseteq H_1\subseteq\dots\subseteq H_m 
        \end{split}\]
    die beiden Normalreihen. Wir setzen nun $G_{ij}:=G_i\cdot(G_{i+1}\cap H_j), H_{ij} := H_j\cdot(G_i\cap H_{j+1}$ für alle $i,j$ mit $G_{n+1}:=G=:H_{m+1}$.
    
    Es gilt nun: $G_{i,m}=G_{i+1}=G_{i+1,0}$ und $H_{n,j}=H_{j+1}=H_{0,j+1}$, so dass $\{G_{ij}\}$ und $\{H_{ij}\}$ Folgen von Untergruppen sind. Es ist klar, dass diese Verfeinerungen von $G$ bzw. $H$ von selber Länge sind. Wir müssen noch zeigen:
    \begin{enumerate}[label=(\alph*)]
        \item $G_{ij}\subseteq G_{i,j+1}$ und $H_{ij}\subseteq H_{i,j+1}$ sind normal, damit beide Normalreihen sind,
        \item $G_{i,j+1}/G_{i,j}$ und $H_{i+1,j}/H_{i,j}$ sind beide isomorph zu $\nicefrac{(G_{i+1}\cap H_{j+1})}{(G_{i+1}\cap H_{j})\cdot(G_{i}\cap H_{j+1})}$.
    \end{enumerate}
    Dann sind die beiden Verfeinerungen äquivalent. Beweis von (a) und (b) erfolgt mit Anwendung der Isomorphiesätze.
\end{proof}

\begin{defn}
    Eine Normalreihe 
        \[\{e\}=G_0\subsetneqq G_1 \subsetneqq\dots\subsetneqq G_{n-1}\subsetneqq G_n=G \]
    heißt Kompositionsreihe, wenn sie keine echten Verfeinerungen besitzt.
\end{defn}

\begin{kor}[Satz von Jordan-Hölder]
    Es sei $G$ eine Gruppe. Zwei Kompositionsreihen in $G$ sind immer zueinander äquivalent.
\end{kor}

\begin{thm}
    Eine Normalreihe ist genau dann eine Kompositionsreihe, wenn die $G_i/G_{i-1}$ jeweils einfache Gruppen sind.
\end{thm}

\begin{proof}
    "$\Rightarrow$": Angenommen, $G_i/G_{i-1}$ sei für ein $i\leq n$ nicht einfach. Es sei dann $\bar{N}\subset G_i/G_{i-1}$ eine nicht triviale normale Untergruppe. Nun setzen wir 
        \[N:=\{g\in G_i: gG_{i-1}\in\bar{N}\}.\]
    Dann ist $N$ eine normale Untergruppe mit $\nicefrac{N}{G_{i-1}}=\bar{N}$. Aus $\{eG_{i-1}\}\subsetneqq\bar{N}\subsetneqq\nicefrac{G_i}{G_{i-1}}$ folgt $G_{i-1}\subsetneqq N\subsetneqq G_i$. Somit können wir $\{G_i\}$ mit $N$ verfeinern, also ist $\{G_i\}$ keine Kompositionsreihe.
    
    "$\Leftarrow$": Wenn $G_0\subsetneqq G_1\subsetneqq\dots\subsetneqq G_n$ eine echte Verfeinerung besitzt, so gibt es eine normale Untergruppe $G_{i-1}\subsetneqq N\subsetneqq G_i$, die $G_{i-1}$ als Normalteiler enthält. Die Faktorgruppe $\bar{N}:=\nicefrac{N}{G_{i-1}}$ ist dann eine nicht-triviale normale Untergruppe von $\nicefrac{G_i}{G_{i-1}}$. Also ist $G_i/G_{i-1}$ nicht einfach.
\end{proof}

\begin{thm}
    Jede endliche Gruppe besitzt eine Kompositionsreihe.
\end{thm}

\begin{proof}
    Übungsaufgabe.
\end{proof}

\begin{ex}

    \begin{itemize}
        \item Unendliche Gruppen besitzen nicht immer Kompositionsreihen. So hat $(\mathbb{Z},+)$ keine Kompositionsreihe, da man jede Normalreihe beliebig verfeinern kann.
        \item Jede symmetrische Gruppe $S_n,\; n\geq 3$ hat die Normalreihe
            \[\{e\}\subsetneqq A_n \subsetneqq S_n.\]
            In dieser Normalreihe haben wir die Faktorgruppen: $\nicefrac{S_n}{A_n}\cong\nicefrac{\mathbb{Z}}{2\mathbb{Z}}$ und $\nicefrac{A_n}{\{e\}}\cong A_n$. Nun ist $\nicefrac{\mathbb{Z}}{2\mathbb{Z}}$ einfach, da die Ordnung eine Primzahl ist, und für $n\neq 4$ ist $A_n$ einfach. Also ist die obige Normalreihe für diese Fälle eine Kompositionsreihe.
    \end{itemize}
\end{ex}

\begin{kor}
    Für $n\geq 3,n\neq 4$ ist $A_n$ die einzige Untergruppe von $S_n$ vom Index $2$. Damit ist $\{e\}\subset A_n\subset S_n$ die einzige Kompositionsreihe und $A_n$ die einzige nicht-triviale normale Untergruppe von $S_n$.
\end{kor}

\subsection{Auflösbare Gruppen}

\begin{defn}
    Eine Gruppe $G$ heißt auflösbar, wenn sie eine abelsche Normalreihe, das heißt eine Normalreihe mit abelschen Faktorgruppen besitzt.
\end{defn}

\begin{ex}
    \begin{itemize}
        \item Abelsche Gruppen sind immer auflösbar.
        \item Die Gruppe $G=\left\{\begin{pmatrix}
            a & b \\
            c & d \end{pmatrix} \in \text{GL}_2(K):c=0\right\}$ ist für jeden Körper $K$ auflösbar.
    \end{itemize}
\end{ex}

\begin{thm}
    Es sei $G$ eine Gruppe.
    \begin{itemize}
        \item Ist $G$ auflösbar, so ist jede Untergruppe von $G$ auflösbar.
        \item Ist $N$ eine normale Untergruppe von $G$, so ist $G$ genau dann auflösbar, wenn $N$ und $\nicefrac{G}{N}$ auflösbar sind.
    \end{itemize}
\end{thm}

\begin{proof}
    (Idee: $H\subset G$ Untergruppe, $G_0\subset G_1\subset\dots\subset G_n$ abelsche Normalreihe für $G$. Dann ist $(G_0\cap H)\subset(G_1\cap H)\subset\dots\subset(G_n\cap H)$ eine abelsche Normalreihe in $H$.)
\end{proof}

\begin{ex}
Die symmetrische Gruppe $S_n$ ist auflösbar für $n\leq 4$. Die Fälle $n\leq 2$ sind sogar abelsch. Für $S_3$ erhalten wir die Kompositionsreihe
        \[\{e\}\subset A_3=\langle(123)\rangle\subset S_3\]
    mit Faktoren $\nicefrac{S_3}{A_3}\cong\nicefrac{\mathbb{Z}}{2\mathbb{Z}}$, $A_3\cong\nicefrac{\mathbb{Z}}{3\mathbb{Z}}$, die beide zyklisch und somit abelsch sind.
    
    Die Gruppe $S_4$ besitzt die Normalreihe
        \[\{e\}\subset V_4\subset A_4\subset S_4\]
    mit $V_4=\left\{\tau\in A_4:\tau^2=e\right\}=\{e,(12)(34),(13)(24),(14)(23)\}$, der kleinschen Vierergruppe. Dann ist $V_4\cong\nicefrac{\mathbb{Z}}{2\mathbb{Z}}\times\nicefrac{\mathbb{Z}}{2\mathbb{Z}}$ abelsch (denn die Ordnung ist $4=2^2=p^2$ für eine Primzahl $p$) und $\nicefrac{A_4}{V_4}\cong\nicefrac{\mathbb{Z}}{3\mathbb{Z}}$, $\nicefrac{S_4}{A_4}\cong\nicefrac{\mathbb{Z}}{2\mathbb{Z}}$ sind auch abelsch.
\end{ex}

\begin{thm}
    Die symmetrische Gruppe $S_n,\;n\geq 5$ ist nicht auflösbar.
\end{thm}

\begin{proof}
    Die Gruppe $A_n$ ist nicht auflösbar, da sie eine nicht abelsche einfache Gruppe ist. Da $A_n$ eine Untergruppe von $S_n$ ist, ist $S_n$ auch nicht auflösbar.
\end{proof}

\end{document}
